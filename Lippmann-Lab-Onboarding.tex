% Options for packages loaded elsewhere
\PassOptionsToPackage{unicode}{hyperref}
\PassOptionsToPackage{hyphens}{url}
%
\documentclass[
]{book}
\usepackage{lmodern}
\usepackage{amssymb,amsmath}
\usepackage{ifxetex,ifluatex}
\ifnum 0\ifxetex 1\fi\ifluatex 1\fi=0 % if pdftex
  \usepackage[T1]{fontenc}
  \usepackage[utf8]{inputenc}
  \usepackage{textcomp} % provide euro and other symbols
\else % if luatex or xetex
  \usepackage{unicode-math}
  \defaultfontfeatures{Scale=MatchLowercase}
  \defaultfontfeatures[\rmfamily]{Ligatures=TeX,Scale=1}
\fi
% Use upquote if available, for straight quotes in verbatim environments
\IfFileExists{upquote.sty}{\usepackage{upquote}}{}
\IfFileExists{microtype.sty}{% use microtype if available
  \usepackage[]{microtype}
  \UseMicrotypeSet[protrusion]{basicmath} % disable protrusion for tt fonts
}{}
\makeatletter
\@ifundefined{KOMAClassName}{% if non-KOMA class
  \IfFileExists{parskip.sty}{%
    \usepackage{parskip}
  }{% else
    \setlength{\parindent}{0pt}
    \setlength{\parskip}{6pt plus 2pt minus 1pt}}
}{% if KOMA class
  \KOMAoptions{parskip=half}}
\makeatother
\usepackage{xcolor}
\IfFileExists{xurl.sty}{\usepackage{xurl}}{} % add URL line breaks if available
\IfFileExists{bookmark.sty}{\usepackage{bookmark}}{\usepackage{hyperref}}
\hypersetup{
  pdftitle={Lippmann Lab Onboarding},
  hidelinks,
  pdfcreator={LaTeX via pandoc}}
\urlstyle{same} % disable monospaced font for URLs
\usepackage{longtable,booktabs}
% Correct order of tables after \paragraph or \subparagraph
\usepackage{etoolbox}
\makeatletter
\patchcmd\longtable{\par}{\if@noskipsec\mbox{}\fi\par}{}{}
\makeatother
% Allow footnotes in longtable head/foot
\IfFileExists{footnotehyper.sty}{\usepackage{footnotehyper}}{\usepackage{footnote}}
\makesavenoteenv{longtable}
\usepackage{graphicx,grffile}
\makeatletter
\def\maxwidth{\ifdim\Gin@nat@width>\linewidth\linewidth\else\Gin@nat@width\fi}
\def\maxheight{\ifdim\Gin@nat@height>\textheight\textheight\else\Gin@nat@height\fi}
\makeatother
% Scale images if necessary, so that they will not overflow the page
% margins by default, and it is still possible to overwrite the defaults
% using explicit options in \includegraphics[width, height, ...]{}
\setkeys{Gin}{width=\maxwidth,height=\maxheight,keepaspectratio}
% Set default figure placement to htbp
\makeatletter
\def\fps@figure{htbp}
\makeatother
\setlength{\emergencystretch}{3em} % prevent overfull lines
\providecommand{\tightlist}{%
  \setlength{\itemsep}{0pt}\setlength{\parskip}{0pt}}
\setcounter{secnumdepth}{5}
\usepackage{booktabs}
\usepackage{amsthm}
\makeatletter
\def\thm@space@setup{%
  \thm@preskip=8pt plus 2pt minus 4pt
  \thm@postskip=\thm@preskip
}
\makeatother

\title{Lippmann Lab Onboarding}
\date{Updated: 2021-06-14}

\begin{document}
\maketitle

{
\setcounter{tocdepth}{1}
\tableofcontents
}
\hypertarget{preface}{%
\chapter*{Preface}\label{preface}}
\addcontentsline{toc}{chapter}{Preface}

The purpose of this document is to outline what the Lippmann lab expects from incoming graduate students and postdoctoral fellows in terms of organization, work ethic, and accountability, as well as to help new lab members get integrated as quickly as possible. Some parts of this document are written in the first person (Ethan's perspective), but many lab members contributed to this document and their voices are heard throughout. We are particularly grateful to Kylie Balotin, who wrote the initial source code to make this document accessible and editable online.

\emph{We acknowledge that the land that Vanderbilt University lies on is the ancestral land of the Cherokee, Chickasaw, Choctaw, Creek, Shawnee, and other Native peoples who were forcefully removed to Oklahoma in the Indian Removal Act of 1830. We honor the elders past and present for the stewardship of this land.}

\hypertarget{general}{%
\chapter{General}\label{general}}

\hypertarget{access}{%
\section{Accessing Lab Accounts and Spaces}\label{access}}

\begin{enumerate}
\def\labelenumi{\arabic{enumi}.}
\item
  The lab has a \href{https://drive.google.com/drive/folders/0Bwvn9S-4oeMmeHhzUS1nM0FNak0?usp=sharing}{Google Drive Folder} that we use for documentation and inventory. It should be kept as up-to-date as possible.
\item
  The lab has a \href{https://lippmann-lab.slack.com/join/signup\#/}{Slack}, which we use for quick messaging and lab-specific threads.
\item
  Request to be added to our lab's listserv by emailing its current handler, \href{mailto:alexander.g.sorets@vanderbilt.edu}{Alex Sorets}. This is generally used for longer, lab-wide announcements.
\item
  Email our current building manager, \href{mailto:ross.denham@vanderbilt.edu}{Ross Denham}, for access to Olin Hall 004A, 117, 118, 119, and 213A, and for afer-hours access to the building. Have Ethan CC'ed on the email so he send his approval to your email.
\item
  Our lab also has 3 calendars to schedule general lab events, book specific hoods, and to book time on our microscope. Request access to these calendars: \href{https://calendar.google.com/calendar/u/0?cid=bXFkOWE3bjM3MTkxbzluYWVtbWw0NDd2bGtAZ3JvdXAuY2FsZW5kYXIuZ29vZ2xlLmNvbQ}{Lippmann Lab Calendar}, \href{https://calendar.google.com/calendar/u/0?cid=ZjNkMXBkbmFwZzc0NWR2YmVmNXU0bGN1YjBAZ3JvdXAuY2FsZW5kYXIuZ29vZ2xlLmNvbQ}{Lippmann Lab Microscope}, \href{https://calendar.google.com/calendar/u/0?cid=MGk3NGZpdm5ua2JyZjMyc2duNTE2aGMxOWdAZ3JvdXAuY2FsZW5kYXIuZ29vZ2xlLmNvbQ}{Lippmann Lab 118 Small Hood}.
\item
  Create an \href{https://vanderbilt.corefacilities.org/landing/2191}{iLabs} account. The first time you log into your account, you will need to add ``Lippmann, Ethan (VU) Lab'' to your groups. You should use the menu on the left side of the screen and click on ``Manage My Groups'' to request access.
\item
  Send an email to the \href{mailto:dts.se.support@Vanderbilt.Edu}{Vanderbilt IT Department} asking for access to our lab's shared network drive. Include your VU NetID in this email.

  \begin{itemize}
  \item
    Mac: \url{smb://vu1file.it.vanderbilt.edu/lippmannlab/}
  \item
    PC: \textbackslash{}\vu1file.it.vanderbilt.edu\lippmannlab\\
  \end{itemize}
\item
  Finally, our lab has a \href{https://github.com/LippmannLab}{Github} that we use for code versioning and internal sharing.
\end{enumerate}

\hypertarget{becoming-an-effective-researcher}{%
\section{Becoming an effective researcher}\label{becoming-an-effective-researcher}}

There is no set formula to facilitate a productive research career but our general expectation is that you will:

\begin{itemize}
\item
  \textbf{Spend 40-60 hours a week in the lab environment.} This includes actual bench work and time at your desk spent updating your lab notebook, organizing data, writing papers, planning new experiments, etc. Learning and innovation in the laboratory is partly through osmosis and casual conversations, so make an effort to be around your peers during normal working hours. You may choose to spend some time outside the lab if you are in the process of writing a paper, but more work is inevitably done inside the lab and student office compared to at home.
\item
  \textbf{Do your work when it needs to be done.} Wet lab work often necessitates experiments at strange hours (e.g.~6 hour time points) or on weekends. Our policy is that you do your work on schedule. Then, if your schedule is free during the week or weekend, take your time off accordingly. I often left work at 3pm during graduate school if I knew I would be pulling a 9-5 on Saturday. Your schedule is your own to set (aside from individual and group meetings) as long as I know you are striving to meet the research goals we have set together. Burnout is a real problem, and we try to avoid it at all costs, but accountability is equally important.
\item
  \textbf{Preparation, preparation, preparation.} Putting together an experimental design by yourself and together with other lab members (especially Ethan) can save days or weeks of unnecessary delays by identifying potential pitfalls, proper control conditions, etc. Always think in great detail about the experiment before you do it!
\item
  \textbf{Be honest with your data - to Ethan and yourself.} Falsifying data is obviously grounds for immediate dismissal from the lab. However, you must also be careful to not bias yourself on what your results actually mean. If you do an experiment three times, and get three different answers, you can't simply report the one that worked the way you thought it should. Always follow the scientific process and determine what you did that gave you these different results. \emph{Your common sense and observational skills are often the most valuable tools in your research.}
\item
  \textbf{Read, read, read.} Our lab pursues diverse research areas -- this is not an accident. All the knowledge I accumulated, which drives the research ideas and grant proposals, was accumulated from years of reading anything I found interesting. If I saw an editorial on an interesting topic, I read it and then read or browsed all the relevant citations associated with it. If I attended an interesting seminar, I thought about the ways we could incorporate that work into our lab's efforts and subsequently browsed papers written by the seminar speaker. However, while I am an excellent resource for both broad and esoteric knowledge, I cannot be your go-to source of information on every topic. Your will have to read primary literature and reviews to build your knowledge base. If you are reading a paper and don't understand a technique, protein, system, etc. that was used, look it up! Wikipedia is a surprisingly helpful for learning biology, and you can find a Youtube video for almost any experimental technique these days. Surprisingly, you can also follow scientists on Twitter who often discuss their recent work or work of their peers, with excellent summaries.
\item
  \textbf{Stay current on the literature.} Set up automatic email alerts from \href{https://www.nature.com/}{Nature}, \href{https://www.sciencemag.org/}{Science}, \href{https://www.cell.com/}{Cell}, and their daughter journals. Then, set up your \href{https://www.ncbi.nlm.nih.gov/myncbi/}{My NCBI account} on PubMed to give you monthly updates on a variety of keywords relevant to your projects. You can use your Vanderbilt account to log into My NCBI by choosing ``See more 3rd party sign in options'' and searching for Vanderbilt. You also access the Vanderbilt library through \href{https://scholar.google.com/}{Google Scholar} by adding all the Vanderbilt Library options to your library links in your Google Scholar settings. The more you know about current findings and techniques that may be relevant to your research, the more novel and innovative you allow yourself to be. This is definitely a skill that should be carried with you after graduate school.
\item
  \textbf{Keep a \protect\hyperlink{zotero}{personal library} of the papers you read.} Use whatever program you want. The majority of the lab uses \href{https://www.zotero.org/}{Zotero} because it's free and compatible with \href{https://www.zotero.org/support/word_processor_integration}{Microsoft Word} for adding a bibliography to a manuscript. This saves a lot of time when trying to track down things you have read and want to cite.
\item
  \textbf{Keep your lab notebook up-to-date.} One of the more annoying events in research is having an experiment fail, then trying to remember what you may have done wrong rather than simply tracing the written details on paper. Your lab notebook should be by your side every day as you work, and you should be writing in it constantly. We buy lab notebooks in bulk from Amazon, so you can get a new one whenever you need it.
\item
  \textbf{Attend campus seminars that are relevant to your research, and probably a few that you simply find interesting.} Hearing a prominent researcher talk for 45 minutes is often more productive than reading 10+ of his/her papers. Likewise, a talk may not be releated to your work, but the techniques being used are. Sign up for email alerts on semiar postings in other departments on campus. Commonly attended seminars include the Chemical Engineering Seminar, Biomedical Engineering Seminar, Stem and Progenitor Cell Interest Group (SPRING) meeting, and the Alzheimer's Disease Journal Club. Furthermore, the number of virtual conferences has also grown in the past year due to the pandemic. Virtual seminars that may be of interest include the CZI Weekly Webinar, \href{https://www.ibbsoc.org/events-calendar/brain-barriers-virtual-2020-bbv2020-seminar-series}{Brain Barriers Virual Seminar Series} and \href{https://www.bme.columbia.edu/tissue-talks-weekly-webinar-series-hosted-dr-gordana-vunjak-novakovic}{Tissue Talks}.
\item
  \textbf{Rely on the experience of others.} Everyone is here to help, particularly given the diverse expertise in our lab. The success of the lab depends on effective teamwork and appropriate integration of new members. When you become the senior lab member, be sure to return the favor to your junior colleagues.
\item
  \textbf{If you want to build something:} The lab has a 3D printer and CNC machine. McMaster-Carr has an amazing catalog of parts, fitting, materials, etc. \href{https://www.vanderbilt.edu/viibre/}{VIIBRE} has an excellent machine shop and offers technical advisement, and we have a small machine shop in the basement of Olin where the CNC is housed. The \href{https://www.vanderbilt.edu/vinse/}{Vanderbilt Institute of Nanoscale Science and Engineering (VINSE)} in the Engineering and Science Building (ESB) has a variety of instruments and a clearn room. Leon Bellan's lab (Mechanical Engineering) has built some interesting new microfluidic capabilities and is an excellent technical resource. Basically, if you think you can design and build something to assist your research, you should. The ability to ``hack'' cell culture and instrumentation is an incredibly useful skill for any career trajectory.
\end{itemize}

\hypertarget{safety}{%
\section{Safety}\label{safety}}

Appropriate personal protection equipment (PPE) should be used. Shorts are fine in the summer, but they must always be accompanied by close-toed shoes. Protective eyeware is required if you are in lab longer than 10 minutes, and especially if you are using any chemicals or the liquid nitrogen. Lab coats are optional in the main lab for routine work but should be donned for chemical handling. Lab coats are required during cell culture. We provide everyone with their own safety glasses/goggles and cell culture lab coat.

\hypertarget{fellowships}{%
\section{Fellowships}\label{fellowships}}

All lab members are expected to apply for external funding. For graduate students, this includes the \href{https://www.nsfgrfp.org/}{NSF GRFP}, \href{https://www.ndsegfellowships.org/about}{NDSEG}, \href{https://researchtraining.nih.gov/programs/fellowships/f31}{NIH F31 fellowship}, and \href{https://professional.heart.org/en/research-programs/application-information/predoctoral-fellowship}{AHA pre-doctoral fellowship}. I am cognizant that certain students are not competitive for certain fellowships, but every student-earned fellowship provides more money that can be spent on supplies and equipment. Current and previous students have uploaded their applications to \href{https://drive.google.com/drive/folders/1j3YPUUUFINandBSU6xtTeN0iOQ3wscMg?usp=sharing}{Google Drive} for internal sharing. There is also a collection of grant examples called the Edge for Scholars (EFS) Funded Grants Library.

\hypertarget{lab-upkeep}{%
\section{Lab Upkeep}\label{lab-upkeep}}

Labs are inevitably messy and disorganized, but there are certain steps that need to be taken to ensure safety is maintained and reagents aren't wasted.

\begin{itemize}
\item
  \textbf{Label the details on everything you make.} What if you pH a buffer to 7.4, but it should have been 8.4? Will you remember these details if your experiment doesn't work and the bottle wasn't labeled? \emph{Also know the shelf life of the reagents you prepare. Expired reagents can cause similar headaches. It is up to you to know how long a particular solution will be active before needing to prepare a new one!} Additionally, please make sure that you put your initials on your reagents when you label them so that other lab members know who to contact if they have questions.
\item
  \textbf{When a product arrives, label it with the date it was received and then opened.} See above.
\item
  \textbf{When the product arrives, make sure it's stored properly.} \href{https://docs.google.com/spreadsheets/d/1t9L23HuwRWHOCMfKu_rz-U7EgKH5gI7dMqieAY7dIXQ/edit?usp=sharing}{Fridges}, \href{https://docs.google.com/spreadsheets/d/1ZrQVxXwKeqDFtX_aHHl4AUQASXxuAAL4-LCeeVI4ZxE/edit?usp=sharing}{-80 Freezer}, \href{https://docs.google.com/spreadsheets/d/1bH7JDGURGhdiMtvVTe1rEuFbch9wmwwFYO38IKKVyBw/edit?usp=sharing}{-20 Freezers}, flammables cabinet, acid/base cabinet, etc.
\item
  \textbf{If you buy a new reagent, add it to our lab \href{https://docs.google.com/spreadsheets/d/1sGEV2wA1jmqhkC78ZvBATzLQRgwiy3oQyAK6NWkWCLE/edit?usp=sharing}{Chemical Inventory}.} Print out its MSDS form and add it to the in-lab binder. Last, add the compound to our \href{https://vanderbilt.bioraft.com/}{Vanderbilt ChemTracker Inventory}. When you sign into the ChemTracker Inventory, you will skip set up, and you should be able to select ``Lippmann Lab'' on the left side of the screen.
\item
  \textbf{Treat reagents appropriately.} Does your reagent need to be desiccated? If so, put it in the right place. Is your reagent anhydrous or hygroscopic? If so, wrap the cap in parafilm to prevent moisture accumulation. And so on\ldots{} If you are unsure how to store, use, or dispose of a reagent, talk to a senior lab member.
\item
  \textbf{Handle antibodies, growth factors, and small molecules appropriately, and always label tubes with dates.}

  \begin{itemize}
  \item
    \emph{Always use sterile microfuge tubes and work in the biosafety hood when preparing anything that will be added to cells!}
  \item
    If an antibody arrives frozen, thaw it on ice and aliquot into individual tubes before refreezing (typically 5 \(\mu\)L per tube).
  \item
    If you buy a new antibody, add it to our \href{https://docs.google.com/spreadsheets/d/1Sga5M7zJor_HN1bLe-AylqsCNEzmls0wY4mIDi0fIvA/edit?usp=sharing}{Antibody Inventory}.
  \item
    Growth factors will typically arrive as lyophilized powders. Reconstitute them according to their instructions (the product sheet should specify water, PBS, dilute HCL, etc.), add 0.1\% human serum albumin, and store as working aliquots in the -80 freezer.
  \item
    \emph{Freeze/thaw of proteins leads to loss of biologyical activity!} Once an antibody or growth factor is thawed, \emph{you should not refreeze it,} but you can store whatever you don't use in the fridge for 1 week. This ensures that these expensive reagents don't get needlessly wasted.
  \item
    Small molecules are typically resuspended in water or DMSO. Check the data sheet before reconstituting and check to see if the reagent is light-sensitive (store in an amber microfuge tube). Then, distribute into microfuge tubes and refreeze. For small molecules in DMSO, if the stock solutions won't be used for \textgreater1 month, wrap the cap of each tube in parafilm to prevent water uptake. If a layer of liquid has formed above the frozen sample, discard it for fear that its activity may be compromised. Additionally, small molecules are typically added to the cells' media the day that you will add it to the cells; exceptions will be noted during training or in the protocols.
  \end{itemize}
\item
  \textbf{Make sure that the liquid nitrogen tanks are filled and its records are current.} Our cell stocks are virtually irreplacable if the liquid nitrogen tanks fail or run dry. Everyone plays a part in making sure their levels are suitably high. Likewise, when you bank cells or remove a vial, be sure to update either the \href{https://docs.google.com/spreadsheets/d/17Vml1k871fifxfslIlFJQ4BDsYGu97dcDBaphiRduKg/edit?usp=sharing}{small tank spreadsheet} or the \href{https://docs.google.com/spreadsheets/d/1hltoHIsF_x7yvHIkptrI5Q1ITN2lx5f3KN6xMcbomQ8/edit?usp=sharing}{big tank spreadsheet} so we know exactly which lines and how many vials we have on hand.
\item
  \textbf{Be a good lab citizen and clean up after yourself.} When you leave the cell culture suite or the lab, there should be no traces that you ever did an experiment. If you use the last of a reagent or are planning on using a significant amount of a reagent, make sure you order more so that you do not affect other lab members' experiments.
\item
  \textbf{The lab works together to make sure the lab spaces are functional, clean, and well-stocked.} We rotate the schedule for \href{https://docs.google.com/document/d/1t22yeyYeN2KZLaDdAZ_DUKlDfBLG7_xTJaLKpJDBCfg/edit?usp=sharing}{dishwashing and autoclaving}. Specific lab duties and equipment liaisons can be also found in \href{https://drive.google.com/drive/folders/14ebHQuo8Vtq63692ZVmemCSDKaYOZeQi?usp=sharing}{this folder}.
\end{itemize}

\hypertarget{values}{%
\chapter{Lab Values}\label{values}}

\hypertarget{lab-mission-statement}{%
\section{Lab Mission Statement}\label{lab-mission-statement}}

Central nervous system (CNS) diseases are a leading cause of death and disability worldwide, and their incidence is expected to increase in tandem with global life expectancies. Many available CNS therapies only ameliorate disease symptoms rather than halting or reversing the actual disease burden, and potential CNS drugs also have lower success rates at all stages of the clinical development pipeline compared to non-CNS candidates. We seek to address this issues through mechanistic investigations, high throughput screening, model system and tool development, and drug delivery.

\hypertarget{code-of-conduct}{%
\section{Code of Conduct}\label{code-of-conduct}}

The lab should be a fun and safe place for everyone to work. Any issues, personal or professional, that affect the lab environment or the well-being of any of its members should be brought to my attention \emph{immediately.} I will do my best to promote a sense of camaraderie and accountability, but it is the responsibility of every individual to be good lab citizens. We do not tolerate poor attitudes and surly behaviors.

We believe that racism, sexism, ableism, transphobia, homophobia, and lack of support for underrepresented minorities prevents our ability to make breakthroughs in science, and we are committed to the creation and maintenance of an equitable and inclusive environment for all of our peers.

\hypertarget{RCR}{%
\section{Responsible Conduct of Research}\label{RCR}}

Vanderbilt and the Lippmann lab are committed to the responsible conduct of research (RCR) as described by the \href{https://oir.nih.gov/sourcebook/ethical-conduct/responsible-conduct-research-training}{NIH}. All first year students are required to take a full day of \href{https://medschool.vanderbilt.edu/bret/responsible-conduct-research/}{RCR training} by the start of their second year. This training is expected to be ongoing every year. Additional RCR courses throughout the year can be found on the BRET office's \href{https://medschool.vanderbilt.edu/bret/seminar-series-calendar/}{calendar} as well as through the \href{https://about.citiprogram.org/en/series/responsible-conduct-of-research-rcr/}{CITI Program}. Breaches in RCR will be addressed seriously and can potential result in dismissal from the lab.

\hypertarget{resources}{%
\section{Resources}\label{resources}}

\hypertarget{graduate-student-and-postdoctoral-fellow-organizations}{%
\subsection{Graduate Student and Postdoctoral Fellow Organizations}\label{graduate-student-and-postdoctoral-fellow-organizations}}

\begin{itemize}
\item
  \href{https://studentorg.vanderbilt.edu/gsc/}{Graduate Student Council}
\item
  \href{https://anchorlink.vanderbilt.edu/organization/bme_gsa}{Vanderbilt Biomedical Engineering Graduate Student Association}
\item
  \href{https://engineering.vanderbilt.edu/chbe/GraduateProgram/CHEGSA.php}{Vanderbilt Chemical Engineering Graduate Student Association}
\item
  \href{https://medschool.vanderbilt.edu/brain-institute/new-vbi-homepage-inprogress/resources-for-students/neuroscience-student-organization/}{Neuroscience Student Organization}
\item
  \href{https://gradschool.vanderbilt.edu/gli/}{Graduate Leadership Institute}
\item
  \href{https://www.vanderbilt.edu/vpa/index.php}{Vanderbilt Postdoc Association}
\item
  \href{https://www.vanderbilt.edu/postdoc/}{Office of Postdoctoral Affairs}
\item
  \href{https://anchorlink.vanderbilt.edu/}{AnchorLink} - good place to look for other student-lead organizations at Vanderbilt
\end{itemize}

\hypertarget{diversity-and-inclusion-organizations}{%
\subsection{Diversity and Inclusion Organizations}\label{diversity-and-inclusion-organizations}}

\begin{itemize}
\item
  \href{https://www.vanderbilt.edu/diversity/contact/}{Office for Equity, Diversity, and Inclusion}
\item
  \href{https://www.vanderbilt.edu/inclusive-excellence/index.php}{Office of Inclusive Excellence}
\item
  \href{https://www.vanderbilt.edu/scsji/}{Student Center for Social Justice \& Identity}
\item
  \href{https://www.vanderbilt.edu/WomensCenter/}{Women's Center}
\item
  \href{https://www.vanderbilt.edu/lgbtqi/}{LGBTQI Life}
\item
  \href{https://www.vanderbilt.edu/bcc/}{Black Cultural Center}
\item
  \href{https://www.vanderbilt.edu/religiouslife/}{Office of University Chaplain and Religious Life}
\item
  \href{https://my.vanderbilt.edu/vuwise/}{Vanderbilt University Women in Science \& Engineering}
\item
  \href{https://gwisnashville.wixsite.com/gwisnashville}{Nashville Chapter of Graduate Women in Science}
\item
  \href{https://www.vanderbilt.edu/lgbtqi/programs/student-groups}{Vanderbilt GoSTEM}
\end{itemize}

\hypertarget{career-development}{%
\subsection{Career Development}\label{career-development}}

\begin{itemize}
\item
  \href{https://medschool.vanderbilt.edu/bret/resources/}{The Office of Biomedical Research Education \& Training}
\item
  \href{https://www.vanderbilt.edu/career/}{Career Center}
\end{itemize}

\hypertarget{mental-and-physical-health-services}{%
\subsection{Mental and Physical Health Services}\label{mental-and-physical-health-services}}

\begin{itemize}
\item
  \href{https://www.vumc.org/student-health/welcome}{Student Health Center}
\item
  \href{https://www.vanderbilt.edu/ucc/}{University Counseling Center}
\item
  \href{https://www.vanderbilt.edu/carecoordination/}{Office of Student Care Coordination}
\item
  \href{https://www.vanderbilt.edu/projectsafe/}{Vanderbilt Project Safe}
\item
  \href{https://www.vanderbilt.edu/studentcarenetwork/}{Student Care Network}
\item
  \href{https://www.vanderbilt.edu/healthydores/}{Center for Student Wellbeing}
\item
  \href{https://gradschool.vanderbilt.edu/current_students/gradlife.php}{Graduate Life Coach}
\end{itemize}

\hypertarget{training}{%
\chapter{Before You Can Work in the Lab}\label{training}}

\hypertarget{logging-onto-oracle-to-access-training-modules}{%
\section{Logging onto Oracle to Access Training Modules}\label{logging-onto-oracle-to-access-training-modules}}

Everyone is required to take several training modules through \href{https://sso-login.vanderbilt.edu/idp/startSSO.ping?PartnerSpId=https://ecsr.login.us2.oraclecloud.com/fed}{Oracle}.

\begin{enumerate}
\def\labelenumi{\arabic{enumi}.}
\tightlist
\item
  Log onto \href{https://sso-login.vanderbilt.edu/idp/startSSO.ping?PartnerSpId=https://ecsr.login.us2.oraclecloud.com/fed}{Oracle}.
\item
  On the home screen, click on the ``Learning'' button.
\item
  Search for the course that you want to take.
\item
  Click on ``enroll'' and complete the class.
\item
  Take a screenshot of the completion screen for each online module and upload it into the lab's \href{https://drive.google.com/drive/folders/1Pbd_SUiZBgL05-BIzgbOpVR8v9ruP0hX?usp=sharing}{training folder}. Additionally, add your information to \href{https://docs.google.com/spreadsheets/d/10eLenI7HV4UPG1CfjAYf9VOY3tXPomWy_ds34bMWDcc/edit?usp=sharing}{this spreadsheet}.
\end{enumerate}

\hypertarget{required-training-modules-for-all-lab-members}{%
\section{Required Training Modules for All Lab Members}\label{required-training-modules-for-all-lab-members}}

\begin{itemize}
\item
  Biosafety 101 - annual retraining required
\item
  Working Safely with Human Derived Specimens - annual retraining required
\item
  Biosafety Principles \& Bloodborne Pathogens Control Topics

  \begin{itemize}
  \tightlist
  \item
    The course schedule can be found \href{https://www.vumc.org/safety/training/biosafety-principles-schedule}{here}
  \item
    To sign up for a class, send your Biosafety 101 completion screenshot to \href{mailto:biosafety@vumc.org}{VUMC Biosafety}. Subject your email with ``Biosafety training request,'' and also include information about which date you would like to sign up for.
  \end{itemize}
\item
  Chemical and Physical Safety in the Lab - renewed annually
\item
  Chemical Waste - renewed annually
\item
  Regulated Medical Waste Shipping Training for Lab Researchers USDOT - renewed every 3-4 years and signed certificate must be kept in lab
\item
  Biosafety Orientation Checklist \& Record - must be filled out by every new lab member and copy kept in lab
\end{itemize}

\hypertarget{required-training-for-lab-members-conducting-animal-studies}{%
\section{Required Training for Lab Members Conducting Animal Studies}\label{required-training-for-lab-members-conducting-animal-studies}}

If you are working with animals, you should also sign up for the following training modules after completing the training listed above:

\begin{enumerate}
\def\labelenumi{\arabic{enumi}.}
\item
  Go to the \href{https://www.vumc.org/acup/}{DAC website}.
\item
  Click \href{https://www.vumc.org/acup/iacuc/training}{training} and log in either as VUMC or VU (depends on your home department).
\item
  This \href{https://www.vumc.org/acup/acup-iacuc-required-course-information}{site} lists all the required training. You are only required to complete sections I and IV.

  \begin{itemize}
  \item
    Section I: Complete all required training for the species you will be working with (most likely mouse) through \href{https://sso.aalaslearninglibrary.org/Shibboleth.sso/Login?entityID=https://sso.service.vumc.org}{AALAS}.
  \item
    Section IV: Register for an ACUP Animal Facility Access Tour.
  \item
    This \href{https://www.vumc.org/acup/acup-iacuc-recommended-courses}{link} also contains some helpful training information.
  \end{itemize}
\item
  Once you complete the required training, send a copy of the transcript to \href{mailto:sarah.m.sturgeon@Vanderbilt.Edu}{Sarah}.
\item
  Fill out the \href{https://www.vumc.org/health-wellness/news-resource-articles/animal-allergy-questionnaire}{animal allergy survey} if not a student or \href{https://www.vumc.org/student-health/sites/default/files/Animal\%20Allergy\%20Questionnaire\%20and\%20IACUC\%20Occ.\%20Health\%20Form.pdf}{this form} if a student (through student health).
\item
  Once complete, let \href{mailto:sarah.m.sturgeon@Vanderbilt.Edu}{Sarah} know, and she will add you to an IACUC protocol.
\item
  Once approved, you will need to sign up for a facility training session. You can ask \href{mailto:sarah.m.sturgeon@Vanderbilt.Edu}{Sarah} for assistance with setting up this training.
\end{enumerate}

For more information, refer to the \href{https://www.vumc.org/acup/welcome}{VUMC Animal Care and Use Program}, email \href{mailto:\%20acuptraining@vumc.org}{ACUP Training}, or ask \href{mailto:sarah.m.sturgeon@Vanderbilt.Edu}{Sarah}.

\hypertarget{meetings}{%
\chapter{Meetings}\label{meetings}}

At the beginning of each semester, I will send out a email with the meeting schedules and locations. I will take into account graduate students' course schedules when assigning the meeting times (if applicable).

\hypertarget{group-meeting-research-presentations}{%
\section{Group Meeting Research Presentations}\label{group-meeting-research-presentations}}

Presentations to your lab mates are often the only practice you will get before having to give a presentation at a seminar or conference. Therefore, I expect these presentations to be taken seriously and reasonably polished. There is no fixed length for a group meeting presentation, but it should have a clear introduction and coverage of the research significance, clear data presentation, and your thoughts on future directions for the work. \emph{Imagine that you are presenting on a manuscript that you are writing: the important parts are the introduction, the methods, the results and data presentation, and your conclusions from the data.} Remember that even if you are presenting similar data from your last presentation, the group may have new members or outside visitors, so it is your job to make sure they understand what you are talking about. This repetition is also incredibly helpful to prepare for presentations in front of strangers, as being on ``autopilot'' for the first few minutes of a talk can be useful if you are nervous. Tips on creating oral presenataions can be found \protect\hyperlink{oralpres}{here}.

\hypertarget{journal-club}{%
\section{\texorpdfstring{\protect\hyperlink{journalclub}{Journal Club}}{Journal Club}}\label{journal-club}}

Every other week, the lab holds a journal club. These meetings consist of presentations on a recent article from the literature that is relevant to our research. When you are in charge of journal club, it is your responsibility to:

\begin{enumerate}
\def\labelenumi{\arabic{enumi}.}
\item
  Choose a manuscript up to a week before the meeting and send it to your lab mates to read and
\item
  Prepare a presentation on this manuscript.
\end{enumerate}

The presentation should have a small amount of background on the PI whose lab conducted the research, a small amount of background on the topic, and the relevant figures from the manuscript that convey its message. Also, be sure to touch on the quality of teh data and whether or not you agree with the conclusions reached or the claims put forth. Lab members who are not presenting are expected to read the article and come prepared to discuss the findings.

\hypertarget{other-group-meetings}{%
\subsection{Other Group Meetings}\label{other-group-meetings}}

Occasionally, other group meetings will be scheduled on the off-weeks of Journal Club to discuss other topics such as Ordering Demos. This information will be discussed when scheduling these additionally meetings. Additionally, Ethan may request for people to attend other meetings for feedback and collaborations.

\hypertarget{individualsubgroup}{%
\section{Individual/Subgroup}\label{individualsubgroup}}

Every other week, lab members will either meet individually or in subgroups based on overlapping research interests. You should always be prepared with the presentation described in the next session and talking points you want to cover, but you shouldn't be afraid to tell me that you couldn't get everything done that you wanted to accomplish or that your experiments failed. Your preparedness for our meetings will often be the primary determinant of how long they last.

\hypertarget{subgroup}{%
\subsection{What to Prepare Before Your Individual/Subgroup Meeting}\label{subgroup}}

Everyone is expected to use the template of:

\begin{itemize}
\item
  What we talked about in the last meeting
\item
  What you said you were going to work on
\item
  What you actually did
\item
  What you're planning to do over the next 2 weeks.
\end{itemize}

Data are expected to be worked up and legible to help ensure that progress is being made towards publications, grants, and graduations. Younger students can ask more senior students for examples if unsure what to prepare for individual/subgroup meetings. Showing up unprepared for a subgroup meeting is incredibly unproductive and will result in extreme annoyance. \emph{Meeting presentations should be stored in \protect\hyperlink{vanderbiltbox}{box} so that we can refer back to them as needed.}

\hypertarget{office-hours}{%
\section{Office hours}\label{office-hours}}

Once a week, I keep an open ``office hour'' when anyone can stop by my office to talk about data, career plans, personal concerns, or any other topic. I am also easily accessible on Slack and by text message.

\hypertarget{practice}{%
\section{Practice Presentations}\label{practice}}

Oral presentations is an essential part of scientific training. If you have any upcoming oral presentations (graduation requirement or general presentation), you can email or Slack message the lab to try to set a time to practice your presentation. Additionally, senior members are happy to look over documents, presentation slides, or posters and provide feedback.

\hypertarget{ordering}{%
\chapter{Ordering}\label{ordering}}

\hypertarget{general-ordering-notes}{%
\section{General Ordering Notes}\label{general-ordering-notes}}

\begin{itemize}
\item
  \textbf{Pay attention to the available amount of disposables.} Many orders take several days for the shipment to arrive. If you use the last pipette or culture plate and we can't get any more for a week, everybody's work suffers! When in doubt, just order multiple boxes of things like pipettes or reorder immediately after you open the last box.
\item
  \textbf{Get quotes on kits or biologics exceeding \$300.} If we buy in bulk, we will almost always get a discount off the website price. All you have to do is email or call the sales department for a company and they will provide a quote to be used on your order. Names, contact information for sales reps, and updated product quotes should be recorded in the \href{https://drive.google.com/file/d/1MHIqY_gJsaVwZvy92M90tAaFEfcg_pQw/view?usp=sharing}{Google Drive Purchasing Folder}. If you are ordering from a new vendor that is not in the Vanderbilt system, you must get a quote and fill in the company information in the purchase request form.
\item
  \textbf{Purchase from \protect\hyperlink{chemstock}{Chem Stock} whenever possible.} Chemicals can be purchased from Fisher but typically the Chem Stock has the same ones at cheaper prices (especially 200 proof ethanol), and if the items are in stock, you can get them right away. You will need to make sure you have the proper secondary container for transferring the chemicals back to lab.
\item
  \textbf{When purchasing from Fisher or VWR, we do not pay shipping.}
\item
  \textbf{Many disposable items that we use are available through the \protect\hyperlink{core}{Molecular Cell Biology Resource Core (MCBR)} - if stocked, these items are avaliable immediately and we do not pay shipping costs.} If not purchasing in bulk, always buy from MCBR if possible. They stock many items from Thermo Fisher Scientific (Life Technologies), Sigma Aldrich, Promega, Qiagen, and Bio-Rad that we use frequently. You can also order non-stocked items through the Core that will NOT include a shipping charge, but you will have to wait for them to ship (a useful option if you don't need these items immediately).
\item
  \textbf{One Purchase Order Form and one requisition must be filled out for each vendor you are ordering from.} You cannot combine multiple vendors on the same purchase order forms. You can split orders from one vendor onto different grants using the same purchase order form.
\item
  \textbf{Do your research if you are buying new antibodies.} We typically can find antibody product information using papers related to our research or by searching in \href{https://www.benchsci.com/}{BenchSci}.
\item
  \textbf{\href{https://www.vanderbilt.edu/skyvu/procurement-requesters.php}{Here} is a helpful website for ordering things through Vanderbilt.}
\end{itemize}

\hypertarget{on-campus-core-facilities}{%
\section{On Campus Core Facilities}\label{on-campus-core-facilities}}

There are a couple places on campus where reagents and chemicals are stocked and can be purchased for immediate use.

\hypertarget{chemstock}{%
\subsection{Chemistry Storeroom (aka ``Chem Stock'')}\label{chemstock}}

Chemicals can be purchased for immediate use at the Chemistry Stock Room (located in the Chemistry building of the Stevenson Center Complex) including 200 proof ethanol, bleach, and acetone. This storeroom also has many other useful items such as NMR tubes, beakers/flasks, gloves, lab notebooks, lab coats, safety glasses/goggles, and tubing. Items can be ordered using the \href{https://vanderbilt.corefacilities.org/sc/4253/vanderbilt-chemical-storeroom/?tab=requests}{iLab Chemistry Storeroom webpage} for pick up. Make sure you have the proper secondary container if transferring chemicals like acetone from the storeroom to the lab. Ethanol and bleach do not need secondary containers.

\hypertarget{core}{%
\subsection{Molecular Cell Biology Resource Core (MCBR, aka ``The Core'')}\label{core}}

Many disposable items that we use are available through MCBR for immediate use including PBS, UltraPure Water, Western Blot secondary antibodies, DMEM/F12, Neurobasal Medium, B27 supplement, N2 supplement, etc. This is the best option if you need a reagent immediately or are not buying in bulk. MCBR is located in rooms 902 and 904 of Light Hall. In addition to stocking supplies, MCBR also provides DNA/RNA oligo and siRNA ordering services, is an instrument facility (rtPCR systems), and a cell culture media preparation service. Purchasing items will be conducted through your iLabs account while you are in the Core. You can also also check whether items are in stock or request for them to be purchased using the \href{https://vanderbilt.corefacilities.org/sc/4722/vumc-mcbr-core-molecular-cell-biology-resource/?tab=services}{iLab VUMC MCBR Core (Molecular Cell Biology Resource webpage)}.

\hypertarget{ordering-training}{%
\section{Ordering Training}\label{ordering-training}}

\hypertarget{departmental-shopper}{%
\subsection{Departmental Shopper}\label{departmental-shopper}}

Departmental shoppers are able to create requisitions in Oracle, but are not able to submit their requisitions for processing. They must reassign their requisitions to a Procurement Requester for the order to be placed. Every lab member should become at least a Departmental Shopper.

\begin{enumerate}
\def\labelenumi{\arabic{enumi}.}
\item
  Access the Introduction to Oracle Cloud Procurement training module in Oracle using the steps listed in the \protect\hyperlink{training}{Training Chapter}.
\item
  Open the ``Me'' section in Oracle.
\item
  Click the ``Roles and Delegations'' button.
\item
  Click the ``Add'' button to request the role of VU\_Departmental Shopper. Click ``Save'' to confirm.
\item
  Make sure that VU\_Departmental Shopper is included in your Role Requests Section. This role should become active within a business day.
\end{enumerate}

\href{https://guidedlearning.oracle.com/player/latest/api/scenario/export/+Rjdn_qARDujqKXlbZWDwA/4jxk6hii/?draft=true}{This link} describes instructions listed above with images.

\hypertarget{procurement-requester}{%
\subsection{Procurement Requester}\label{procurement-requester}}

Procurement Requester are able to create and submit requisitions in Oracle.

\begin{enumerate}
\def\labelenumi{\arabic{enumi}.}
\item
  Email \href{mailto:f.c.baquera@VANDERBILT.EDU}{Felisha} to ask her to put you on the list for the purchasing training course in Oracle. Let her know if you are already a Departmental Shopper and that you would like to become a Procurement Requester. At least a few lab members should become a Procurement Requester.
\item
  Complete the online training, which includes meeting with Felisha or the FUM.
\item
  Attend a training session conducted by Purchasing Services.
\item
  Have someone in the lab review your first catalog and non-catalog orders.
\end{enumerate}

\hypertarget{how-to-place-orders}{%
\section{How to place orders}\label{how-to-place-orders}}

Commonly purchased items are listed in \href{https://docs.google.com/spreadsheets/d/13m76wa6D7RpNgtVolgLW7oUlB0JmVAlQNCRtXGdosbc/edit?usp=sharing}{this spreadsheet}. This \href{https://drive.google.com/file/d/1MHIqY_gJsaVwZvy92M90tAaFEfcg_pQw/view?usp=sharing}{Purchasing Folder} also stores additional useful ordering information.

\hypertarget{prior-to-creating-a-requisition-in-oracle}{%
\subsection{Prior to Creating a Requisition in Oracle}\label{prior-to-creating-a-requisition-in-oracle}}

\begin{enumerate}
\def\labelenumi{\arabic{enumi}.}
\item
  Download and fill out the \href{https://drive.google.com/file/d/1sv-9x9ky6_tleJ2rFb5CJseFnJjDlw88/view?usp=sharing}{Purchase Request Form}.

  \begin{itemize}
  \item
    Required information includes: Vendor, Justification, POET Numbers, Item Quantity, Unit, Model/Catalog \#, Description, Unit Cost, and Total Cost , Quote Number (if applicable).
  \item
    If splitting order among many grants, list this information in the in the COA/POET number(s) section.
  \item
    Justification should include information about what you are purchasing and how it will be used for the grant you are charging it to.
  \item
    General lab items can have a justification such as ``{[}Item{]} will be used for general lab procedures/cell culture for {[}grant name{]}.''

    \begin{itemize}
    \tightlist
    \item
      Amazon orders must also include this statement : ``These items could not be found with these specifications or this price from any other vendor.''
    \end{itemize}
  \item
    Save a copy of this file for your records. The recommended naming format is ``Lippmann lab purchase request form - YearMonthDay - Vendor.xlsx.'' Make sure to save your quotes too for future reference.
  \item
    You can also check the Aquiire catalog for the prices of your items, as these prices are often lower than on the normal vendor websites.
  \item
    One purchase request form should be filled out for each vendor you are purchasing from.
  \end{itemize}
\item
  Send the PO request forms (and quotes if applicable) to Ethan for approval. Make any edits that he lists and resend for final approval. Save his approval emails for later.
\item
  Once you have Ethan's approval, you can proceed to creating a requisition in Oracle by selecting the ``Purchase Requisitions'' button in the Procurement tab.
\end{enumerate}

\hypertarget{catalog-orders}{%
\subsection{Catalog Orders}\label{catalog-orders}}

Catalog orders refer to any vendor that can be found on the Aquiire catalog. These vendors include Abcam, Cell Signaling Technology, Fisher, Guy Brown, McMaster, Sigma, IDT, and VWR. A list of the current vendors in the Aquiire catalog and additional tips can be found \href{https://drive.google.com/file/d/1FoXTe82I5IBqeLMY3-MZ39DXdckFvCIQ/view?usp=sharing}{here}.

\begin{enumerate}
\def\labelenumi{\arabic{enumi}.}
\item
  Click on the down button next to ``Shop by Category'' and select ``Aquiire.''
\item
  Select the vendor you are ordering from.

  \begin{itemize}
  \tightlist
  \item
    Make sure that the vendor is linked to your Oracle Cloud account before proceeding. The page usually lists your email or notes the Vanderbilt Purchasing Department on the home page.
  \item
    If you are not linked, simply return to the Aquiire catalog page and reselect the vendor.
  \end{itemize}
\item
  Add the items you have been approved to purchase.
\item
  Go to your cart and checkout. This should return you to the Aquiire Catalog.

  \begin{itemize}
  \item
    If you have a promo code for your order, you should input this information on the vendor website before checking out.
  \item
    For many quotes for Aquiire vendors, you can enter the quote number before you check out. For Fisher, you can search for your quote by going to ``Your Account'' at the top right and selecting ``Quotes'' in the drop-down menu. If a vendor does not have a way to enter quote information through the catalog, you will need to follow the \protect\hyperlink{noncatalog}{non-catalog order instructions}.
  \end{itemize}
\item
  Click the ``Check Out'' button on the Aquiire Catalog page, which will create the beginning of your requisition.
\item
  For the ``Description'' section, include this information at the minimum: Your Last Name - Lippmann - Vendor. You can also include information about what you are ordering in this requisition if you want.
\item
  Copy and paste your justification from your purchase request form to the ``Justification'' section.
\item
  Under attachments at the top right section of the Requisition, include a copy of Ethan's approval. Make sure the Category is set to ``Approver.'' Click ``OK'' to save. See the \protect\hyperlink{reassign}{``Information to Include When Reassigning to a Procurement Requester''} section for additional attachments to include if you are a Departmental Shopper.
\item
  Make sure your ``Deliver-to Location'' is set to Olin Hall 107.

  \begin{itemize}
  \item
    The exception is AL Gas Orders.
  \item
    You can make this your default by clicking the pencil next to your name in the ``Requisitions'' page. You can then change your ``Deliver-to Location'' to Olin Hall 107, and then click on ``Save and Close.''
  \end{itemize}
\item
  Each item needs to have the Billing information filled out. You can edit multiple items by selecting all the items that go on the same grant, clicking the down arrow by ``Actions,'' and selecting ``Edit.''

  \begin{itemize}
  \item
    Under ``Project Number,'' put the grant number before the period (ex: 601413 for CZI).
  \item
    Under ``Task Number,'' put the grant number after the period (ex: 1 for CZI or 10 for Start-Up).
  \item
    The ``Expenditure Type'' should be ``SuppLab'' for almost every requisition you make.
  \item
    The ``Expenditure Organization'' is 15220 - Chemical Engineering. This is true for all requisitions you make no matter what your home department is since Ethan's primary appointment is in the Chemical Engineering Department.
  \item
    You should fill out the ``Charge Account'' with the following information. You can edit this section by clicking the button directly to the right of the text box.

    \begin{itemize}
    \item
      Entity: 150 - School of Engineering
    \item
      NetAssetClass: 15 - Sponsored Contract and Grants (except Start Up which is 10 - Unrestricted Faculty and Student Organization Funds)
    \item
      FinancialUnit: 15220 - Chemical Engineering
    \item
      Account: 6105 - Supplies Expense - Lab
    \item
      All other section should be kept at the default (000 or 0). These are just potential future labeling numbers that are not currently in use.
    \end{itemize}

    -Click ``OK'' when done filling out.
  \item
    ``Budget Date'' should be the date you are submitting the requisition.
  \item
    If adding multiple items to the same billing information, put 100\% in the ``Percentage'' section. If splitting the same item across multiple grants, send to a Purchase Requester to fill out as this can get slightly complicated.
  \end{itemize}
\item
  You can save your requisition by hitting the ``Save'' button. It is recommended that you do this often. If you want to save the requisition to work on later, you can click the down arrow next to ``Save'' and click on ``Save and Close.''
\item
  If you are a Purchase Requester, you can hit the ``Submit'' button when you are done editing a requisition. If you are a Departmental Shopper, you can reassign your requisition to a Purchase Requester by clicking on ``Save and Close.'' Then you will need to click the requisition to reopen it, select the down arrow next to ``Actions,'' and choose ``Reassign.'' You can search the Purchase Requester's name, and make sure you check ``Send notifications to this person'' so that they know you have transferred a requisition for them to check and submit.
\end{enumerate}

\hypertarget{al-gas-orders}{%
\subsubsection{AL Gas Orders}\label{al-gas-orders}}

We purchase liquid nitrogen, carbon dioxide, and nitrogen gas from AL gas. For gas orders, you will need to include our account number 1218 and the quantity of tanks being delivered/removed from the room number in your purchase order form. You should use the search function on the left of the Aquiire page to find the gas tanks needed and add to cart (you don't use the AL Gas catalog). You should change the delivery location in your requisition from Olin 107 to the room the gas tanks need to be delivered to. In the requisition's ``Notes to Supplier'' field and the ``Justification'' field, you should write the following information: ``Please deliver to Olin Hall {[}room number{]}. Gas account \#1218.'' Also include the number of tanks that you would like removed from the lab space if swapping out multiple tanks. \textbf{\emph{If you change out a tank or empty a tank, you should put in an order for a new tank immediately.}} The liquid nitrogen tank should be ordered when you notice that the tank is almost empty or if you have emptied the tank.

\hypertarget{noncatalog}{%
\subsection{Non-Catalog Orders}\label{noncatalog}}

Non-catalog orders refer to vendors that are not on the Aquiire catalog. This includes Peprotech, Genscript, Ted Pella, and \protect\hyperlink{addgene}{Addgene}. There are four different types of non-catalog requisitions, but we almost always use the mon-catalog order by quantity and rarely use the non-catalog order by amount. Non-catalog order by quantity should be used if you're buying a specific number of an item (ex. 2 incubators, 5 tweezers) and non-catalog order by amount should be used for things like service/maintenance.

\begin{enumerate}
\def\labelenumi{\arabic{enumi}.}
\item
  Under the ``Request Forms'' sections of the Requisition page, select ``Non-Catalog Order by Amount.'' This will bring you to a page where you fill out the information for \emph{one} item that will be on this order.
\item
  Under ``Supplier,'' search for your vendor.
\item
  If applicable, provide a part number in the ``Item/Part Number'' section. Otherwise, just put ``N/A.'' For orders containing multiple items, we will add each item one at a time to our cart.
\item
  Under ``Item Description,'' write any information that might be helpful in helping the Purchasing Department place the order with the vendor. For example, if you are ordering a peptide, make sure you include the sequence information in this section. You will also want to place the item name and part number (if applicable) in this section as well. Make sure this information matches your quote and/or the vendor's website.
\item
  For the ``Category'' section, you will usually select ``Lab Supplies'' or ``Lab Equipment.'' The exception is if you are purchasing software, such as BioRender, which is considered ``IT Computer Software.'' Additionally, if an item is more than \$5,000, you must go through the capital equipment process.
\item
  For UOM name, you will usually select EA (each), and add quantity information in the item description section. For ``Amount,'' you will add the cost of each item based on your quote or the vendor website.
\item
  Add your item to your cart. This will create the requisition with the item you just added, but will NOT save it. Navigate to the cart by clicking the blue number next to the shopping cart icon in the top right, then selecting ``Review.''
\item
  To add additional items, make sure the requisition line of the item you added is selected, then go to actions -\textgreater{} duplicate. This will create an identical line, line item 2. Select the line, then go actions -\textgreater{} edit. Fill out the item description, category name, quantity, UOM Name, Price, and Item/Part Number sections for the new items, then hit ``ok'' to save changes. If this item is on a different grant, you will need to edit the grant information as well. Repeat for each item.
\item
  You can either save your requisition to work on late by clicking the ``Done'' button. Otherwise, you can continue to work on the requisition by clicking on the cart button at the top right of the screen.
\item
  Fill out the requisition form the same way you would fill out a catalog order requisition form. Make sure you include the quote number in your description name (if applicable). Additionally, make sure you include the following list of attachments:

  \begin{itemize}
  \item
    To Buyer: Attach Quote (if applicable)
  \item
    To Supplier: Attach Quote (if applicable)
  \item
    To Approver: Attach Quote (if applicable)
  \item
    To Approver: Attach Ethan's approval email
  \end{itemize}
\item
  Submit your order by reassigning it or submitting it yourself.
\end{enumerate}

\hypertarget{addgene}{%
\subsubsection{Addgene Orders}\label{addgene}}

In order to order products from \href{https://help.addgene.org/hc/en-us/articles/205436319?_ga=2.100504030.1196238464.1608166799-1984471690.1608166799\&_gac=1.53248474.1608166799.CjwKCAiA_eb-BRB2EiwAGBnXXjNbFJq4bB11oAb4wVAgpFjnxn-sVv-PdYz6QgEBRQ9U0g-_hKce1BoCTW0QAvD_BwE}{Addgene}, you need to greate an account. Once you receive your purchase order (PO) from Purchasing Services, you can add your items to your online cart and choose ``Pay by Purchase Order'' when you are checking out. If you do not enter your order in Addgene yourself, your order will not be placed, and your items will never arrive.

\hypertarget{amazon-and-capital-equipment-orders}{%
\subsection{Amazon and Capital Equipment Orders}\label{amazon-and-capital-equipment-orders}}

Send your Amazon order purchase request form and Ethan's approval email to the \href{mailto:chbeorders@vanderbilt.edu}{Chemical Engineering Department}. The subject of the email should be ``Your Last Name - Lippmann - Vendor.'' Make sure your justification includes this statement in your purchase request form: ``These items could not be found with these specifications or this price from any other vendor.'' Capital equipment is when at least one component is greater than \$5,000. If you are unsure, talk to \href{https://finance.vanderbilt.edu/purchasingandpaymentservices/purchasingservices/}{Purchasing Services}. These steps should be followed for any vendor that uses credit cards for ordering rather than Oracle Requisitions.

\hypertarget{new-vendors-to-vanderbilt}{%
\subsection{New Vendors to Vanderbilt}\label{new-vendors-to-vanderbilt}}

If a vendor is not listed in Aquiire or when searching in the supplier section of a non-catalog order:

\begin{enumerate}
\def\labelenumi{\arabic{enumi}.}
\item
  Double check that the supplier is actually not listed. Many vendors use a different name. When searching in Oracle, be sure to check both ``Supplier'' and ``Alternate Name.''
\item
  If the vendor is not a registered supplier with Vanderbilt, you will need to start the process, which can take a few weeks. Only purcurement requestors, not departmental shoppers, can do this process. You can follow the \href{https://www.vanderbilt.edu/skyvu/procurement-requesters.php}{``New Supplier Requests'' instructions}.
\item
  You will need to ask for the following information from the vendor:

  \begin{itemize}
  \item
    an IS Form W-9,
  \item
    a copmleted, signed ACH/Bank form (copy on the procurement requestor website),
  \item
    a contact name, address, phone number, and email for a vendor employee. You will need to enter this information into Oracle. When you enter Oracle and navigate to the page where you can select the tile ``Purchase Requisitions'', there is another tile next to it listed ``Suppliers''. Select the Suppliers tile, then on the right sidebar click the tab above the magnifying glass tab, then select ``Register Supplier'' and fill out the information. You will be notified when the request has been completed and the supplier is registered with Vanderbilt. You will then be able to place non-catalog orders with them.
  \end{itemize}
\end{enumerate}

\hypertarget{reassign}{%
\subsection{Information to Include When Reassigning to a Procurement Requester}\label{reassign}}

When you are reassigning your requisition to a Procurement Requester, please include the following information:

\begin{itemize}
\item
  Purchase request spreadsheet
\item
  A copy of Ethan's approval
\item
  Quote (if applicable)
\end{itemize}

Additionally, please check the box to send a notification to the Procurement Requester so that they know it has been reassigned to them. Procurement Requesters will approve it as quickly as they can so that the order gets placed.

\hypertarget{information-to-include-when-asking-a-lab-member-to-order-your-supplies}{%
\subsection{Information to Include When Asking a Lab Member to Order Your Supplies}\label{information-to-include-when-asking-a-lab-member-to-order-your-supplies}}

Sometimes lab members will send out a Slack message to see if anyone wants to add items to their orders. If you want them to order something for you, please include the following information:

\begin{itemize}
\item
  Name of item
\item
  Catalog number
\item
  Vendor (if asking them to create a new requisition for your supplies)
\item
  Quote (if applicable)
\item
  Quantity of item you want ordered
\item
  Link to website (optional, but helpful if not a commonly purchased item)
\end{itemize}

\hypertarget{approving-orders}{%
\subsection{Approving Orders}\label{approving-orders}}

For some vendors, Purchasing Services will send you an email to check that you were charged the correct amount and that the items arrived. When you get this email, you can either approve it in Oracle or you can approve the email directly (click the approve button and send an empty email). Make sure you do this in a timely manner because if there are too many orders that are awaiting approval, Vanderbilt and/or the vendor may halt all future orders (from any lab) until the outstanding approvals are answered.

\hypertarget{commonly-used-grants}{%
\section{Commonly Used Grants}\label{commonly-used-grants}}

These are commonly used grants to purchase lab supplies. Ethan may direct you towards more project specific grants depending on your work.

\begin{itemize}
\tightlist
\item
  Discetionary accounts: FF\_220114.10 and FF\_220114.30

  \begin{itemize}
  \tightlist
  \item
    Generally don't use unless buying capital equipment or Ethan instructs you to use it. The Start-Up grant is also used to purchase office supplies.
  \end{itemize}
\item
  CZI: 601413.1

  \begin{itemize}
  \tightlist
  \item
    For activities that are specific to BBB transporter biology or don't ``fit'' on other grants (pseudo-discetionary money).
  \end{itemize}
\item
  R01: SFP\_300234.1

  \begin{itemize}
  \tightlist
  \item
    For most cell culture supplies, disposables, and general wet lab reagents.
  \end{itemize}
\end{itemize}

\hypertarget{datamanagement}{%
\chapter{Data Management and Shared Resources}\label{datamanagement}}

This section will describe how the Lippmann Lab stores and manages data.

\hypertarget{storage-and-documentation}{%
\section{Storage and Documentation}\label{storage-and-documentation}}

\hypertarget{lab-shared-network-drive}{%
\subsection{Lab Shared Network Drive}\label{lab-shared-network-drive}}

Our lab has a shared network drive folder where members of the lab can store raw data. You should have requested access to this folder in the \protect\hypertarget{access}{}{Accessing Lab Accounts and Spaces} section of this document. The address of this folder is \url{smb://vu1file.it.vanderbilt.edu/lippmannlab/}. In order to log in, you will need to type ``VANDERBILT/Your NetID'' for the username and your normal Vanderbilt password for the first password. If it is your first time logging in, you should set up a folder with your name. This is a communal resource, so everyone has access to your files once you've uploaded them to the drive. Make sure you are saving your files to this location from the Leica microscope or you are using your own external drive as storage space on the Leica computer is limited. Some members of the lab also use this as backup for important documents such as qualifying exams, fellowship applications, etc.

\hypertarget{other-shared-access-drives}{%
\subsubsection{Other Shared Access Drives}\label{other-shared-access-drives}}

Many \protect\hyperlink{corefacilities}{core facilities} also have their own folders for storing and managing data. You can check each core facility's website to find the address for its specific folder. Additionally, the Bellan lab has it's own folder for saving confocal data; you will need to ask a Bellan lab member for access to this remote folder.

\hypertarget{remote-access-using-pulse-virtual-private-network-vpn}{%
\subsubsection{Remote Access Using Pulse Virtual Private Network (VPN)}\label{remote-access-using-pulse-virtual-private-network-vpn}}

If you want to access data saved in these folders off-campus, you will need to set up \href{https://it.vanderbilt.edu/services/catalog/end-point_computing/network_access/remote-access/index.php}{Pulse VPN}. Note that you will need \protect\hyperlink{duo}{Duo Multi-Factor Authentification} set up to use the Pulse VPN service. If you are on campus and connected to the Vanderbilt WiFi, you should be able to open these folders without connecting to the VPN.

\hypertarget{googledrive}{%
\subsection{Google Drive}\label{googledrive}}

The \href{https://drive.google.com/drive/folders/0Bwvn9S-4oeMmeHhzUS1nM0FNak0?usp=sharing}{lab Google Drive folder} is another storage site commonly used to share information with the entire lab. This information includes locations of reagents, protocols, and example fellowship/qualifying exam documents. Lab members generally do not use this space to store data. You should be updating these files if you move or receive reagents or if you develop any protocols for the lab.

\hypertarget{vanderbiltbox}{%
\subsection{Vanderbilt Box}\label{vanderbiltbox}}

\href{https://vanderbilt.account.box.com/login}{Vanderbilt Box} is where most data and document sharing with Ethan occurs. You should use your normal VU NetID and password to log into this account. If this is your fist time logging in, you should make a folder entitled ``{[}Your Name{]} Research'' and share it with Ethan. This is where research updates, subgroup meeting presentations, raw paper data, etc. is stored. \textbf{\emph{Once you have submitted a paper, make sure that Ethan has all the raw data in a Box folder in order to comply with RCR standards!}} \protect\hyperlink{subgroup}{Subgroup meeting presentations} and lab meeting presentations should also be stored in your box folder for future reference.

\hypertarget{github}{%
\subsection{GitHub}\label{github}}

The \href{https://github.com/lippmannlab}{Lippmann Lab GitHub} is where code sharing occurs. There are different repositories for different programs that lab members use to quantify their data or control their bioreactors. Generally, the repositories that involve quantifying data stores a README file that explains how to install and use the code, all the code needed to run the program, and an example file to run the code with. If you edit any part of these repositories, make sure your commit messages are detailed enough that others can understand what changes you have made. If you are creating a new repository, follow the ``Neuron Image Processor'' as a template for what should be included. Public repositories such as the Spinfinity code does not require a GitHub account to access, but any private repositories will require that you create an account and that a lab member adds you as a collaborator to the project.

\hypertarget{computer-back-ups}{%
\subsection{Computer Back Ups}\label{computer-back-ups}}

You should be backing up your computer relatively frequently. This can take many forms but can be extremely useful if your computer crashes or you accidentally delete data.

\hypertarget{zotero}{%
\section{Reference Management}\label{zotero}}

Most members of the lab use \href{https://www.zotero.org/}{Zotero} as their reference manager. You should \href{https://www.zotero.org/download/}{download} the program to your computer at the beginning of your graduate training. Additionally, it is helpful to install the Chrome connector, which allows you to easily save papers that you are reading to Zotero. Furthermore, Zotero has plugins for both \href{https://www.zotero.org/support/word_processor_plugin_usage}{Word} and \href{https://www.zotero.org/support/google_docs}{Google Docs} that will allow you to insert both citations and bibliographies effortlessly into papers. Zotero allows you to format your citations and bibliographies in a variety of different citation styles, and you can easily convert your settings in the same document.

Another free reference manager that some lab members use is \href{https://www.mendeley.com/download-desktop-new/}{Mendeley}, although it is much easier to have everyone using the same program when writing papers in groups.

Furthermore, this \href{https://www.youtube.com/watch?v=Y40Wy1guAiQ}{Excel Journal Database} may also be a good tool for keeping up with what papers you have read during your training, although it should not be used as your citation manager.

\hypertarget{communication}{%
\section{Communication}\label{communication}}

This section will contain tips on how to present scientific work. An entire chapter is dedicated to writing manuscripts.

\hypertarget{oralpres}{%
\subsection{Creating an Oral Presentation}\label{oralpres}}

All graduate students and postdocs will present a research update once a semester at the lab group meeting. Additionally, lab members may have the opportunity to speak at department retreats, seminars, and conferences.

\begin{itemize}
\item
  \textbf{Make sure you don't put too much content on each slide.} The goal is for your audience to listen to you as you speak and not spend the entire time trying to read your slide. Figures are preferable over text. Large or complicated figures should be broken down. You can use basic animations to make the information contained in the slides, but don't get too cute with your use of animations. You can also split complicated topics into a couple slides. Animations or pointers can also help your audience know what you are discussing on each slide.
\item
  \textbf{Make sure your slide titles are descriptive and contain the main takeway.} Don't use slide titles that don't add any information or is too technical.
\item
  \textbf{Citations should be enough information for you audience to find the articles/websites.} This is usually the fist author, journal title, and year at the minimum for articles. You should cite all figures that are not originally made by you.
\item
  \textbf{Make sure your text is large enough to be seen from far way.}
\item
  \textbf{Figues should be fully explained.} You should describe what the data points are, what the axes mean, etc. in addition to explaining the main takeway from the figure.
\item
  \textbf{Remember that anything you put on your slide is fair game.} The audience has the right to ask you questions on anything that you put on your slide. This is especially true for qualifying exams and dissertation defenses. Be careful putting in information or figures that you are not prepared to explain in detail.
\item
  \textbf{It's usually a good idea to pause briefly before moving on to the next slide in order to see if anyone has any questions.} You can also incorporate planned pauses into your presentations in order to make room for audience questions.
\item
  \textbf{Be aware of who your audience is when you are creating your presentations.} The way that you present in a lab meeting might be different than how you present at a conference. Make sure you know who might be in your audience when you are putting it together so that you don't get too technical and detailed or too broad and superficial. This includes knowing when you should define abbreviations. For example, you might not need to define iPSCs or BBB in lab meetings, but you should define it at a qualifying exam or at a conference.
\item
  \textbf{If you are using a pointer, make sure you practice using it before you present.}
\item
  \textbf{\protect\hyperlink{practice}{Practice} your oral presentation with the lab!} If you are presenting at a conference/meeting or for a department milestone, it is highly encouraged that you practice with the lab. Just send out an email or Slack message about availability, and senior students would be happy to watch a practice presentation.
\end{itemize}

\hypertarget{journalclub}{%
\subsubsection{Journal Club Presentations}\label{journalclub}}

The lab rotates who presents an article for our lab's journal club, which is usually emailed out at the beginning of each semester. There are also other opportunities to present at other journal clubs on campus such as at the Alzheimer's Disease Journal Club. In addition to the tips from the above section on oral presentations, here are some additional tips for presenting at journal clubs.

\begin{itemize}
\item
  \textbf{Picking an article to present:} You should choose a paper that has been published relatively recently. Ethan will typically assist younger students before their first presentation. These articles should be sent to lab at least a week prior to presenting.
\item
  \textbf{Know background material about the chosen paper.} You should have knowledge of the research and work that contributed to the paper you chose to present. You should provide enough information in the introduction for lab members who may not be familiar with the topics described in the paper. Simplify complex topics for non-experts to understand.
\item
  \textbf{Describe the research question and the importance/context of the research.} This can also help organize your presentation as the authors draw conclusions to answer these research questions.
\item
  \textbf{Break down figures and other complex information.} The way data is presented in a paper is not necessarily the way it should be presented in an oral presentation. Break down figures into smaller, understandable chunks so that it's not too overwhelming to your audience. Additionally, make sure you make the figures large enough for your audience to read. Be ready to discuss all the components of the figures. Try not to make your slides too text heavy.
\item
  \textbf{Provide information on uncommon techniques used in the paper.}
\item
  \textbf{Be ready to discuss the paper in lab.} Identify pros and cons about the techniques used and/or the conclusions drawn from the paper. What questions did you still have after reading the paper? This helps show an understanding of the paper, and what you learn from this practice can help with peer reviewing papers. Additionally, it can lead to better lab discussions about the paper.
\item
  \textbf{Be ready to discuss why you chose that paper.} You can talk about how it relates to your work or techniques you are interested in using.
\end{itemize}

\hypertarget{creating-a-poster-presentation}{%
\subsection{Creating a Poster Presentation}\label{creating-a-poster-presentation}}

Lab members may also present their work in poster at internal and external meetings and conferences.

\begin{itemize}
\item
  \textbf{Know who your audience is when creating your poster.} This will help determine how much background information needs to be provided as well as what jargon you should use. Your poster should not be too technical and detailed if you are presenting to a broad conference, but it should also not be too broad or simplified if you are presenting at a field-specific conference.
\item
  \textbf{Your posters should contain at minimum:} an introduction/background/significance section, at least one results/discussion/methods section, a conclusion/future directions section, an acknowledgments section, and a reference section.
\item
  \textbf{Figures and diagrams are preferable to text, but you should give enough information that your audience can understand your figures and the conclusions that are drawn from your results.}
\item
  \textbf{Use bullet points with short sentences/phrases rather than paragraphs of text for easier reading.} Make sure any text is concise and relevant. Use consistent formatting throughout the poster. Try to avoid redundancy and vague phrases. Make sure your results headings draw attention to the question you are answering and/or conclusions you are drawing from the results.
\item
  \textbf{Make sure figures and text are legible from far away.}
\item
  \textbf{Use blank space wisely.} This empty space can help draw attention to important information. Scale and color can also demonstrate relative importance.
\item
  \textbf{Practice how you are going to present your poster to your audience.}
\item
  \textbf{Senior students would be happy to look over posters and provide feedback.}
\item
  Poster templates can be found \href{https://drive.google.com/drive/folders/1V1ykFkQy5sY_4hgSlZhh6y-axiy3mreH?usp=sharing}{here}. You can download these templates and edit it for your own posters.
\item
  You can refer to the \href{https://www.insidehighered.com/news/2019/06/24/theres-movement-better-scientific-posters-are-they-really-better}{Better poster movement} for additional tips on creating a poster presentation.
\end{itemize}

\hypertarget{computationdata-analysis}{%
\section{Computation/Data Analysis}\label{computationdata-analysis}}

\hypertarget{accessing-software}{%
\subsection{Accessing Software}\label{accessing-software}}

We use a variety of software for analyzing data. \href{https://www.r-project.org/}{R} and \href{https://rstudio.com/products/rstudio/download/}{RStudio} are open source and free to download. \href{https://imagej.net/Fiji}{Fiji/ImageJ} is an open source tool for processing images. Other sofware such as Microsoft Office and Matlab can be downloaded for free from \href{https://it.vanderbilt.edu/software-store/}{Vanderbilt IT Services}. Other software can also be bought at a discount through Vanderbilt IT. You can discuss any software needs with Ethan in order to see if the lab should buy a license. \href{https://my.vanderbilt.edu/cisr/microscopy-resources/downloads/}{CISR} also has some tools that can be downloaded for free for image processing and visualization.

\hypertarget{norms-for-analysis}{%
\subsection{Norms for Analysis}\label{norms-for-analysis}}

\begin{itemize}
\item
  Our lab uses \href{https://drive.google.com/file/d/1eRxUonvqSrtfm6mePRuh6x_J_fUgkLIo/view?usp=sharing}{this Excel spreadsheet} as a template for analyzing TEER data.
\item
  Make sure scale bars are sufficiently large enough to be seen if the image is small. You should have a scale bar in at least one image if you have a series of images that are at the same scale.
\end{itemize}

\hypertarget{documentation}{%
\subsection{Documentation}\label{documentation}}

\begin{itemize}
\tightlist
\item
  \textbf{Protocol generation and storage occurs in our lab's \protect\hyperlink{googledrive}{Google Drive Folder}.}

  \begin{itemize}
  \item
    Make sure you cite any papers that your protocols might be based off of.
  \item
    Make sure you note any edits you make to existing protocols.
  \item
    All protocols should contain a list of reagents (with product numbers) and reagent preparation/storage information in addition to the protocol.
  \item
    Include graphics as necessary to understand the timeline of different differentiation protocols. You can also include markers that are used to verify cell identities as applicable.
  \item
    Add a sufficient amount of information in your protocol so that your lab members can follow the protocol with minimal guidance.
  \end{itemize}
\item
  \textbf{Code documentation is stored in our lab's \protect\hyperlink{github}{Github}.}

  \begin{itemize}
  \item
    Make sure your commits are informative and explain what you changed and why.
  \item
    Make sure you provide an example of how to run your code.
  \item
    Make sure you describe how to install your code and what the inputs/outputs should be.
  \end{itemize}
\end{itemize}

\hypertarget{tools}{%
\subsection{Tools}\label{tools}}

\begin{itemize}
\item
  \textbf{Learning to Code}

  \begin{itemize}
  \item
    \href{https://chanzuckerberg.com/ndcn/member-login/?redirect_to=https://chanzuckerberg.com/ndcn/}{CZI} holds weekly \href{https://sifuentescj.gitbook.io/comp-bio-training/}{Computational Biology Office Hours} to address computational needs in the neurodegeneration community. They can also hold Hackathons to specifically address any computational problems that a lab might want to solve. The CZI Neurodegeneration website also has a section where you can access any computational resources created in the community.
  \item
    \href{https://www.vanderbilt.edu/datascience/events/data-science-workshops/}{Vanderbilt Data Science Institute Workshops}
  \item
    \href{https://www.library.vanderbilt.edu/disc/workshops.php}{Digital Scholarship and Communications Education and Training}
  \item
    \href{https://openscapes.github.io/cyop/}{Choose youR own Pythway}
  \item
    \href{https://rstudio-conf-2020.github.io/r-for-excel/}{R for Excel Users}
  \item
    \href{https://r4ds.had.co.nz/}{R for Data Science}
  \item
    \href{https://www.edx.org/course/statistics-and-r}{Statistics and R}
  \item
    \href{https://jakevdp.github.io/PythonDataScienceHandbook/}{Python Data Science Handbook}
  \item
    \href{https://hellowebbooks.com/learn-command-line/}{Really Friendly Command Line Intro}
  \item
    \href{https://software-carpentry.org/lessons/}{Carpentries Sofware Lessons} and \href{https://datacarpentry.org/lessons/}{Carpentries Data Lessons}
  \end{itemize}
\item
  \textbf{Learning about sc-RNAseq Analysis}

  \begin{itemize}
  \item
    \href{https://scrnaseq-course.cog.sanger.ac.uk/website/index.html}{Analysis of single cell RNA-seq data}
  \item
    \href{https://github.com/theislab/single-cell-tutorial}{Current best-practices in single-cell RNA-seq: a tutorial}
  \item
    \href{https://support.10xgenomics.com/single-cell-gene-expression/software/pipelines/latest/using/tutorial_ov}{Getting Started with Cell Ranger (10X Platform)}
  \item
    \href{https://satijalab.org/seurat/articles/get_started.html}{Getting Started with Seurat}
  \end{itemize}
\item
  \textbf{Learning How To Conduct Literature Reviews}

  \begin{itemize}
  \item
    \href{https://researchguides.library.vanderbilt.edu/researchproductivityworkshops}{Research Productivity Workshop Series}
  \item
    \href{https://researchguides.library.vanderbilt.edu/digitalliteraciesworkshopseries}{Digital Literacies Workshop Series}
  \end{itemize}
\item
  \textbf{Learning about Image Processing}

  \begin{itemize}
  \item
    \href{https://imagej.net/Introduction}{Intro to ImageJ}
  \item
    \href{https://my.vanderbilt.edu/cisr/microscopy-resources/downloads/}{CISR Downloads Page}
  \item
    \href{https://benchfly.com/video/5781/basics-of-scientific-digital-imaging/}{Basics of Scientific Digital Imaging}
  \item
    \href{https://benchfly.com/video/5496/basics-of-fiji-imagej-part-1/}{Basics of FIJI/ImageJ Part 1}
  \item
    \href{https://benchfly.com/video/5605/basics-of-fiji-imagej-part-2/}{Basics of FIJI/Image J Part 2}
  \item
    \href{https://www.dropbox.com/sh/0k4z8z8hpofd5a0/AABPEkTbxkLzWVxsmpKNkVSja?dl=0}{FIJI Demo Files}
  \end{itemize}
\end{itemize}

\hypertarget{manuscript-preparation}{%
\chapter{Manuscript Preparation}\label{manuscript-preparation}}

Your papers are your currency from your time in the lab - peer-reviewed proof that you are are highly skilled and successful researcher. As such, the final versions should be clear and concise but thorough and polished. Read a few papers Ethan published during graduate school and his postdoctoral fellowship (\href{https://stemcellsjournals-onlinelibrary-wiley-com.proxy.library.vanderbilt.edu/doi/full/10.1002/stem.1622}{Lippmann et al, \emph{Stem Cells,} 2014}; \href{https://www-nature-com.proxy.library.vanderbilt.edu/articles/nbt.2247}{Lippmann et al, \emph{Nat Biotechnol,} 2012}) to get an idea of the style he prefers. Additionally you can see a list of the papers that have been published by members of the lab \href{http://www.lippmannlab.com/publications}{here}.

\hypertarget{the-process}{%
\section{The Process}\label{the-process}}

\begin{enumerate}
\def\labelenumi{\arabic{enumi}.}
\item
  \textbf{Outline the paper.} In general, I prefer that you meet with me to discuss an outline of the paper, most notably the general message of the introduction and how the data/figures will support the claims made, prior to initiating the writing process. It is common to think about your existing data, figure sets, and missing pieces before starting the writing process.
\item
  \textbf{Pick a target journal.} For obvious reasons, not everything gets submitted to Nature or Science. Have a good idea for what you think the significance of your work is, whether it belongs in a broad journal versus a specialty journal, and how quickly you want to get it published.
\item
  \textbf{Read the journal's author guidelines.} Most journals have their own specific figure guidelines and word limits. Unless you want to tear your hair out reformatting, check the journal requirements first.
\item
  \textbf{You will write the first draft - Ethan will be heavily involved in your first paper-writing attempt but preferably less so thereafter.} Some re-writing by me is inevitable, but I try to be constructive if I don't agree with the language you've used or the conclusions you've drawn. The first draft typically gets marked up a lot, and subsequent edits are much faster.
\item
  \textbf{Track changes as best you can between versions.} If you don't agree with a change I have made, make a comment in the document so I don't think you're just ignoring my opinion. Save each version as a different file in case you need to go back to something written in an earlier version.
\item
  \textbf{Assemble your figures in Photoshop or Illustrator.} Most journals require a certain resolution and may have their own requirements on fonts and sizes. I typically do all of my figure preparations, including labels, in PowerPoint and then assemble these parts in Photoshop at the desired figure resolution (typically 300-400 dpi). After assembly, images can be compressed by flattening the layers into one final plane. Most journals also require images to be below a certain size (usually 10MB).

  \begin{itemize}
  \item
    It is becoming standard practice to include individual data points in figures like bar graphs in order to show data distribution.
  \item
    The lab has a \href{https://biorender.com/}{Biorender} license, so you can ask Ethan to get access to the program in order to make figures.
  \item
    \href{https://it.vanderbilt.edu/software-store/index.php}{Vanderbilt IT Software Store} provides free software to students, staff, and faculty including Microsoft Office 365 ProPlus and ChemBioDraw as well as discounted subscriptions for programs like Adobe Photoshop and Prism. If you think a program would be broadly beneficial for the lab to purchase a license for, discuss it with Ethan.
  \end{itemize}
\item
  \textbf{After assembling the figures and submitting the paper, provide Ethan with the figures} \textbf{\emph{minus}} \textbf{their lettering (a, b, c, etc.).} This makes it easier to use these figures for presentations.
\item
  \textbf{Any data used to compile a graph of some sort should be provided to Ethan in an Excel spreadsheet.} This ensures that he has a copy of the original numbers after you leave, complying with \protect\hyperlink{RCR}{RCR practices}. Many students choose to set a folder for their manuscripts in \href{http://vanderbilt.box.com/}{Box} that contains raw and processed data.
\item
  \textbf{\emph{Once you have submitted a paper, make sure that Ethan has all the raw data in a Box folder in order to comply with RCR standards!}}
\item
  \textbf{Celebrate when the paper is accepted.} First round for the entire lab is always on Ethan! It is important to revel in the success of your colleagues.
\end{enumerate}

\hypertarget{writing-resources}{%
\section{Writing Resources}\label{writing-resources}}

\begin{itemize}
\item
  The university has a few writing accountability programs including \href{https://www.vanderbilt.edu/healthydores/writers-accountability-group-wag-workshop/}{WAG}.
\item
  The university also provides a \href{https://www.vanderbilt.edu/writing/}{writing studio} where you can get feedback on your manuscripts. The BRET office lists additional \href{https://medschool.vanderbilt.edu/postdoc/vanderbilt-resources/scientific-writing/}{resources} that are available for improving scientific writing.
\item
  Dr.~Katja Brose from CZI also offers office hours where you can get advice on papers and publishing. The CZI Newsletter includes information about how to sign up to meet with her.
\end{itemize}

\hypertarget{frequently-asked-questions}{%
\chapter{Frequently Asked Questions}\label{frequently-asked-questions}}

These are topics that are commonly asked about when new members join the lab. Additionally, feel free to ask senior lab members for more information or advice whenever you have questions.

\hypertarget{scheduling-vacations}{%
\section{Scheduling Vacations}\label{scheduling-vacations}}

Taking time off is necessary to avoid burn-out during graduate school. While iPSC culture requires daily maintenance, it is still possible to take vacations and rest. Short trips over the weekend do not need to be cleared with me, but I should be aware if you are planning on taking longer periods of time off. I have yet to tell any lab members that they could not take a vacation, but you need to be smart about when you plan to be away so that progress is not disrupted.

If you are asking another lab member to cover your work while you are out of lab, you should ask that lab member at least a week before you are gone (unless it's a personal health or family emergency). Before you leave you should provide the following information:

\begin{itemize}
\item
  How long you'll be out of lab
\item
  What cells need their media changed each day you are gone and what media to use following this \href{https://docs.google.com/document/d/1e7WmtnTOwzWEQ0M4DY7jF22D8st5nZTYDcJ1Tv7Nfsg/edit?usp=sharing}{template}.
\item
  Where the cells are located
\item
  Where media and other necessary reagents are located
\item
  How to make more media if the lab mate is not used to maintaining the cell types
\item
  How to operate devices (if needed)
\item
  If you want more iPSCs when you come back and if you are leaving Matrigel plates
\end{itemize}

You should not:

\begin{itemize}
\item
  Ask your lab mate to run crucial experiments for you while you are gone
\item
  Maintain short term differentations (ex: BMECs) or common iPSCs (ex: CC3s) if you'll be gone for a week or more
\end{itemize}

Typically, first year students will take Thanksgiving break off. More senior students will rotate who covers this holiday. The lab generally shuts down for the holidays in December/January, although a couple of lab members may stay behind for maintaining long-term cultures (like neurons or organoids) or animals. Holidays such as the 4th of July or Labor Day are usually treated as half days where many people usually choose to do their bare minimum work and take the rest of the day off. If lab meetings fall on the U.S. Election Day, Ethan will cancel those meetings in order to allow lab members to vote if they have not already.

Overall, it is up to you to make sure that you are getting the necessary work done in order to graduate on time.

\hypertarget{mental-health-in-graduate-school}{%
\section{Mental Health in Graduate School}\label{mental-health-in-graduate-school}}

Mental health is a major issue in graduate school. Many members of the lab use mental health resources available on campus including the \href{https://www.vanderbilt.edu/ucc/}{University Counseling Center (UCC)} and the \href{https://gradschool.vanderbilt.edu/current_students/gradlife.php}{Vanderbilt Graduate Life Coach}. Our medical insurances provides free access to \href{https://www.vanderbilt.edu/ucc/services/psychiatric-services/}{psychiatry services} although medication may still cost some money. There are also outside resources such as \href{https://www.phdbalance.com/}{PhD Balance}. Feel free to reach out to other members of the lab if you are having issues or need people to talk to! Several students over the years have been up front with me about their mental health struggles and I have made efforts to accommodate their needs, but I can only help if I know there is a problem. I fully acknowledge that I sometimes ask students to do too much, so if you are struggling or overly stressed because of lab work, it needs to be discussed so we can come to a mutual decision on acceptable outputs.

\hypertarget{advice-on-navigating-co-mentorships}{%
\section{Advice on Navigating Co-Mentorships}\label{advice-on-navigating-co-mentorships}}

\begin{itemize}
\item
  No matter whether Ethan is your primary or secondary adviser, you are welcome to use the Lippmann lab spaces and equipment (some equipment such as the biosafety hoods, microscope, and osmometer needs extra training). You are also invited and encouraged to attend Lippmann lab social events.
\item
  Ideally, your co-advisor will be a faculty member that Ethan already has a working relationship with (a close collaborator or someone on a shared grant)
\item
  Time spent in the labs of both mentors is rarely split evenly, so make sure it is clear to everyone who will serve as your ``primary'' mentor, why this choice was made, and the relative time that you will be spending in each lab.
\item
  At the beginning of your co-mentorship, make sure to have a meeting where all three of you discuss the general outline of your research direction. While this may be difficult to do during your first semester, make sure that you are actively communicating with Ethan and your other advisor about where your project is at and any research changes. You are responsible for communicating between your advisors and making sure that they are on the same page with the status of your research. Co-mentorship is meant to enhance the scope of research but it can get messy if communication breaks down.
\item
  Similarly, if you're feeling overburdened with double lab meetings, journal clubs, etc, then talk with Ethan about it! The goal of a co-advising relationship is for you to have the freedom to develop a more interdisciplinary project by pulling on the expertise of Ethan and your co-advisor. While it might require more time-juggling on the front end (learning two lab cultures, attending double lab meetings, doing more background reading to get acquainted), make sure that you're also not neglecting your research! Communicate, communicate, communicate. It will always help you to be clear with Ethan about your stress level/workload and you can find a solution by having open conversations about it.
\end{itemize}

\hypertarget{registering-for-courses-in-the-yes-portal}{%
\section{Registering for Courses in the YES Portal}\label{registering-for-courses-in-the-yes-portal}}

Graduate students register for courses on the \href{http://yes.vanderbilt.edu/}{Vanderbilt YES Portal}. You can register for courses using the ``Student Registration'' tab. All graduate students are required to enroll in 9 credit hours every semester.

\begin{itemize}
\tightlist
\item
  Enrolling in Classes:

  \begin{itemize}
  \tightlist
  \item
    Search for the courses you are required to take/interested in taking
  \item
    Click on the course and click the ``Add to Cart'' button at the bottom of the pop up screen.
  \item
    Move the to ``In Cart'' tab.
  \item
    Click the down arrow at the left side of the course. Click on ``Enroll.''
  \item
    You can see how many courses/credit hours you are enrolled in for each semester by moving to the ``Enrolled'' tab.
  \end{itemize}
\item
  Enrolling in Research Credit:

  \begin{itemize}
  \tightlist
  \item
    Search your home department in the search bar.
  \item
    If you have not passed your qualifying exam yet, select Ethan's course in the Non-Candidate Research offerings. If you have passed your qualifying exam, select Ethan's course in the Ph.D.~Dissertation Research offerings.
  \item
    Enroll in the course using the instructions listed in the ``Enrolling in Classes'' bullet point.
  \item
    Move to the ``Enrolled'' tab. On the right side of the course, there is an edit button (looks like a pencil and paper). Click the edit button.
  \item
    Select the number of research credit hours you need to enroll in, and click the ``Save'' button.
  \item
    This needs to be done every semester even after you are done taking classes for Vanderbilt tuition purposes (Ph.D.~students do not pay for tuition themselves).
  \end{itemize}
\end{itemize}

\hypertarget{recommended-elective-courses}{%
\section{Recommended Elective Courses}\label{recommended-elective-courses}}

Graduate students are expected to take courses that are required by their specific department. Some training grants such as the T32 in Environmental Toxicology also have course requirements that appointees must take. The number of elective classes you can take varies by department, but here's a list of common elective courses that Lippmann Lab graduate students take to finish their course requirements:

\begin{itemize}
\item
  NURO 8345: Fundamentals of Neuroscience I (offered every Spring)
\item
  NURO 8365: Neurobiology of Disease (offered every Spring)
\item
  CHBE 5890: Special Topics - Biomolecular Engineering and Design (Ethan's elective course that is offered every other Spring)
\item
  BME 8901: Special Topics - Mol,Cellular,\&TissueMechanobio (offered every Fall)
\item
  BME 7410: Quantitive Methods in Biomedical Engineering (offered every Fall)
\item
  CHBE 5820: Immunoengineering (offered every Spring)
\item
  BME 8901: Computational Genomics (offering variable)
\end{itemize}

BME students can take 1 Chemical Engineering Course as as BME Course requirement with no questions asked. Additional courses must be cleared by \href{mailto:cynthia.reinhart-king@VANDERBILT.EDU}{Dr.~Reinhart-King}.

\hypertarget{corefacilities}{%
\section{Core Facilities on Campus}\label{corefacilities}}

There are a variety of equipment that graduate students and postdocs can use through \href{https://www.vanderbilt.edu/cores/cores-facilities.php}{Vanderbilt University (VU)} and the \href{https://www.vumc.org/oor/institutional-research-shared-resources-and-core-facilities}{Vanderbilt University Medical Center (VUMC)}. Common core facilities that the Lippmann Lab uses includes the VUMC Flow Cytometry Shared Resource, VUMC Translational Pathology Shared Resource, VU NMR Facility, VU Cell Imaging Shared Resource (CISR), and the Vanderbilt Institute of Nanoscale Science and Engineering (VINSE). If you are interested in seeing what equipment the lab is trained to use, you can look at \href{https://docs.google.com/spreadsheets/d/1gsTkMHcHbQ4z3y_Iq0_OSRODjhGSvkoRmYmUvWKNb2M/edit?usp=sharing}{this spreadsheet}. You can also talk to Ethan about getting trained to use different facilities on campus. Students should have an iLabs account set up in order to use most core facilities.

\hypertarget{potential-vanderbilt-funding-sources}{%
\section{Potential Vanderbilt Funding Sources}\label{potential-vanderbilt-funding-sources}}

There is the potential to have travel, research, and your stipend funded from Vanderbilt sources other than Ethan's grants at Vanderbilt. This is a (non-comprehensive) list of potential funding sources open to Lippmann Lab trainees.

\hypertarget{graduate-student-council-gsc-travel-grant}{%
\subsection{Graduate Student Council (GSC) Travel Grant}\label{graduate-student-council-gsc-travel-grant}}

GSC offers up \$500 \href{https://studentorg.vanderbilt.edu/gsc/travel-funding-request/}{travel awards} for enrolled Vanderbilt graduate students. Graduate students are eligible for travel grants up to 3 times during their training.

\hypertarget{graduate-leadership-institute-gli-grants}{%
\subsection{Graduate Leadership Institute (GLI) Grants}\label{graduate-leadership-institute-gli-grants}}

GLI offers two types of funding opportunities: \href{https://gradschool.vanderbilt.edu/gli/deg/}{GLI Dissertation Enhancement Grants} and \href{https://gradschool.vanderbilt.edu/gli/pdt/}{GLI Professional Development and Training Grants}. The Dissertation Enhancement Grants offers up to \$2,000 for dissertation research expenses. All Ph.D.~students in good academic standing are eligible to apply. The Professional Development and Training Grants offers up to \$1,000 for professional development and training opportunities. This can include traveling to conferences. All Vanderbilt graduate students in good academic standing are eligible for this grant.

\hypertarget{vanderbilt-institute-for-clinical-and-translational-research-victr-vouchers}{%
\subsection{Vanderbilt Institute for Clinical and Translational Research (VICTR) Vouchers}\label{vanderbilt-institute-for-clinical-and-translational-research-victr-vouchers}}

The \href{https://victr.vumc.org/}{Vanderbilt Institute for Clinical and Translational Research} can be a source of funding to use core facility equipment on campus. Ethan will usually talk to you about applying for VICTR vouchers, and resources for filling out the application can be found in \href{https://drive.google.com/drive/folders/1xe3LeIbXhmq09adtJ0cnFZ08d9W6Qn29?usp=sharing}{this folder}. Note that especially motivated undergraduate students can also apply for vouchers. If you receive an updated biosketch from Ethan, please upload it to the folder. You can make a biosketch using the SciENcv section of \href{https://www.ncbi.nlm.nih.gov/myncbi/}{your My NCBI account}.

\hypertarget{training-grants}{%
\subsection{Training Grants}\label{training-grants}}

There are several NIH-funded training grants that can cover up to 2 years of your training. Lippmann lab members have been funded through the T32 Training Grant in Alzheimer's Disease and the T32 Training Grant in Environmental Toxicology. The activities required while funded by the training grants will vary but can include presenting at a seminar, participating in a journal club, or taking certain courses. Typically, Ethan will talk to you about training grant funding opportunities, and applications usually require a CV.

\hypertarget{common-conferences}{%
\section{Common Conferences}\label{common-conferences}}

While Ethan is always open to suggestions of conferences you would like to present at, here is a list that many lab members have attended in the past:

\begin{itemize}
\item
  American Chemical Engineering Society's Annual Meeting
\item
  Biomedical Engineering Society's Annual Meeting
\item
  Society for Neuroscience's Annual Meeting
\item
  Chan-Zuckerberg Initiative Neurodegeneration Challenge Network Annual Meeting
\item
  Gordon Research Conference: Barriers of the CNS
\item
  Vanderbilt Annual Alzheimer's Disease Research Day
\item
  Vanderbilt NanoDay
\end{itemize}

\hypertarget{the-office-of-biomedical-research-education-and-trainings-bret-career-resources}{%
\section{The Office of Biomedical Research Education and Training's (BRET) Career Resources}\label{the-office-of-biomedical-research-education-and-trainings-bret-career-resources}}

The BRET office has \href{https://medschool.vanderbilt.edu/career-development/}{career development resources} for both graduate students and postdocs. The \href{https://medschool.vanderbilt.edu/career-development/aspire-bistro-for-phd-students/}{Bistro series} is focused on graduate students, while the \href{https://medschool.vanderbilt.edu/career-development/aspire-cafe-for-postdoctoral-fellows/}{cafe series} is focused on post doctoral fellows. Additionally the BRET office has a \href{https://medschool.vanderbilt.edu/career-development/aspire-job-search-series-for-phd-students-and-postdocs/}{job search series} and a career fair that are open to graduate students and postdocs.

\hypertarget{early-milestones-in-each-department}{%
\section{Early Milestones in Each Department}\label{early-milestones-in-each-department}}

There's a lot of variability in the department graduation requirements. Here's a short summary of some of the milestones that occur up to the qualifying exam and what is required for each milestone for the departments that are commonly represented in the Lippmann Lab. Graduate school documents for forming your thesis committee, scheduling your qualifying exam, and preparing your dissertation/defense can be found \href{https://gradschool.vanderbilt.edu/academics/forms_timeline.php}{here}. Make sure you are aware of your department's specific requirements for each milestone.

\hypertarget{chbe}{%
\subsection{Chemical and Biomolecular Engineering (CHBE)}\label{chbe}}

\begin{itemize}
\item
  \textbf{Departmental Exam:} This exam typically occurs in late August/early September. Students are expected to prepare a 10 page written report and a 20 minute oral presentation. Students should focus on how the principles of chemical engineering applies to their research and should demonstrate the progress they have made on their research projects -- the latter point is the most important, while the former point will typically be raised by the faculty during the Q\&A. One retake is held in December if students do not pass the first exam.
\item
  \textbf{Qualifying Exam:} Chemical Engineering students are expected to take their qualifying exam by the end of their 5th semester at Vanderbilt (although most students take it sometime during their 2nd year). Students should prepare a written document and an oral presentation on their dissertation proposal. Written documents should be given to the students' committees at least 2 weeks before the oral component of the exam. This exam can be taken a maximum of 2 times.

  \begin{itemize}
  \tightlist
  \item
    \textbf{Thesis Committee:} A minimum of 4 people including Ethan, 2 CHBE faculty members, and an additional faculty member from outside the department. Students should meet with their committee at least once a year after passing their qualifying exam.
  \end{itemize}
\end{itemize}

\hypertarget{bme}{%
\subsection{Biomedical Engineering (BME)}\label{bme}}

\begin{itemize}
\item
  \textbf{Mentoring Committee Meetings:} Prior to forming our thesis committee, BME graduate students are randomly assigned a committee of 3 BME professors (including Ethan) around November of their first year. BME students are expected to meet with their mentoring committee every 6 months until they form their thesis committee. These meetings are typically just 30 minutes with 15 minutes spent presenting to your committee and 15 minutes for questions. The template for mentoring committee meeting presentations can be found \href{}{here}.
\item
  \textbf{Qualifying Exam:} Lippmann lab BME students typically take their qualifying exam during their 3rd year at Vanderbilt (although there is some flexibility). Students should prepare a written document and an oral presentation on their dissertation proposal. Written documents should be given to the students' committees at least 2 weeks before the oral component of the exam.

  \begin{itemize}
  \tightlist
  \item
    \textbf{Thesis Committee:} A minimum of 5 people including Ethan, 2 BME faculty (with at least one having a primary appointment in BME), at least one member from VUMC, and an additional member from any department. BME students are expected to meet with their committee at least every 6 months after passing their qualifying exam.
  \end{itemize}
\end{itemize}

\hypertarget{ims}{%
\subsection{Interdisciplinary Materials Science Program (IMS)}\label{ims}}

\begin{itemize}
\item
  \textbf{Preliminary Exam:} This exam will be taken before the end of the third semester at Vanderbilt, ideally after the majority of required coursework is completed. The main purpose of this exam is to confirm that you have the potential to make a successful PhD student and that you have an adequate understanding of the foundations of materials science from all previous courses. Since our lab is mostly concerned with biomaterials, this exam will be focused on a short presentation of all the research you have accomplished (10-15 min) and an ``area paper'' (4-10p) which will need to be submitted to your committee at least one week before your exam. You will be responsible for forming a preliminary exam committee which consists of five faculty members (including Ethan) that are from at least three different departments. This may be the same committee that you choose to move forward with for your qualifying exam, but it does not have to be. If you fail your first examination, your committee may decide to allow you to sit for a second exam or suggest that you complete the additional research work required to earn a master's degree.
\item
  \textbf{Qualifying Exam:} The IMS qualifying same is similar to quals for other departments. Its main purpose is to determine that your proposed research will be worthy of attaining your PhD. It must be completed within four years of entering Vanderbilt and at least six months before defending. Similar to the Preliminary Exam, there is both a written and oral portion although each are more lengthy. The written proposal should be sent to your committee at least two weeks before your exam and is a formal, detailed justification of your proposed hypothesis future research direction, supported by the work that you have completed so far. You will then give a 30-minute talk to your committee, followed by a period of oral examination.

  \begin{itemize}
  \tightlist
  \item
    \textbf{Thesis Committee:} The composition of your thesis committee for your qualifying exam and PhD defense may be the same as for your preliminary exam. Your thesis committee should comprise of your 2 thesis advisers and 3 other faculty members. Three faculty must be members of the Graduate Faculty and at least 3 affiliated with the Materials Program. For more detailed information see ``Expectations for IMS preliminary and qualifying exams'' document that you received during orientation.
  \end{itemize}
\end{itemize}

\hypertarget{igp}{%
\subsection{Interdisciplinary Graduate Program (IGP) and Quantitative and Chemical Biology (QCB): Neuroscience Track}\label{igp}}

\begin{itemize}
\item
  \textbf{Qualifying Exam:} All neuroscience students take their qualifying exams at the end of the second year. The requirements typically include a written exam in the form of a Nature Neuroscience style review document. This review is expected to cover literature that supports the student's disseration research topic and must be broadly related to their thesis. Alongside the review document, the student is expected to submit a specific aims page outlining their plan for the next few years of grad school (work on this begins as part of a grant writing course in the Spring). Neither documents can be reviewed by the PI and must be original work. Following submission of the written portion of the exam, students take an oral qualifier comprised of a 5 minute chalk talk followed by open question by the committee for up to 2 hours. This includes topics like general knowledge of neuroscience, experimental design and thesis project. Oral exams usually occur between June to August. The qualifying exam committee must comprise of three professors of the student's choice and one department appointed committee chair.

  \begin{itemize}
  \tightlist
  \item
    \textbf{Thesis Committee:} The composition of the thesis committee may differ from that of the qualifying exam committee. These are typically professors with research backgrounds similar to the student's thesis project. You are required to have 2 Neuroscience faculty members and 1 external faculty member on your committee. Committee meetings should typically occur atleast once a year until the final defense or more frequently as requested by the student or committee.
  \end{itemize}
\end{itemize}

\hypertarget{advice-on-scheduling-committee-meetings}{%
\section{Advice on Scheduling Committee Meetings}\label{advice-on-scheduling-committee-meetings}}

It can be notoriously difficult to schedule meetings with committee members due to faculty members' busy schedules. We recommend choosing a 2 week period of time and confirming that Ethan will be available those weeks. You can also ask your committee members if there are days that are generally not good for them to meet (ex: their lab meeting schedules or days they teach). Once you have a shortened list of potential days and times, you can create a Doodle poll to determine what will work for all your committee members. Note that if you have a VUMC faculty member on your committee, they may have secretaries that you can email for quicker responses.

\hypertarget{czi-resources}{%
\section{CZI Resources}\label{czi-resources}}

The Chan-Zuckerberg Initiative (CZI) Neurodegeneration Challenge Network (NDCN) provides resources for Lippmann lab trainees. They hold an annual conference where Ethan can take 2 trainees; he will discuss how trainees will be selected for this conference before it starts. They also hold a weekly seminar series. The schedule for this seminar will be sent out in the monthly newsletter. All seminar presentations are recorded in the \href{https://ndcn-forum.cziscience.com/}{NDCN forum}. Additionally, there is a Slack workspace that staff, graduate students, and postdocs can access. There are weekly Computational Office Hours where trainees can learn to a variety of computational topics including processing RNAseq data and creating reproducible code. There is also a staff and trainee database being developed for further collaborations and networking. Talk to Ethan or senior lab members if you are interested in getting more involved with the opportunities provided by CZI.

\hypertarget{duo}{%
\section{Two Factor Authentification}\label{duo}}

Vanderbilt uses Duo Security for extra protection when logging onto Vanderbilt website and to access files remotely. Use \href{https://it.vanderbilt.edu/services/catalog/security/identity_and_access_management/Multi_Factor_Authentication.php}{this link} to set up Duo with your computer or phone.

\hypertarget{onetab}{%
\section{OneTab}\label{onetab}}

\href{https://chrome.google.com/webstore/detail/onetab/chphlpgkkbolifaimnlloiipkdnihall?hl=en}{OneTab} is a Google Chrome extension that some lab members have found useful. When you inevitably have billions of tabs open in your browser, OneTab can be used to condense all the tabs into a single list that you can organize.

\hypertarget{mentoring-opportunities}{%
\section{Mentoring Opportunities}\label{mentoring-opportunities}}

There are many opportunities to mentor younger students in the Lippmann lab.

\begin{itemize}
\item
  \textbf{High School Students:} Ethan is involved with a partnership with a local high school that brings high school students on campus to learn about (and potentially participate in) research. He will contact you if there is an opportunity to mentor a high school student. Typically, these students will shadow you while you conduct your research and submit an article to the \href{https://www.youngscientistjournal.org/}{Vanderbilt Young Scientist Journal}. This is can range for a couple months to a couple years commitment. In addition to research, some students may also ask you questions about college and the application process. Occasionally, other high school students will spend a day shadowing you in lab.
\item
  \textbf{Undergraduate Graduate Students:} There are many opportunities to mentor undergraduate students whether they are Vanderbilt students or visiting summer students. Talk to Ethan if you are interested in mentoring an undergrad or could use help with your projects.
\item
  \textbf{Younger Graduate Students:} Senior graduate students are expected to help mentor and train younger graduate students. Typically, students sharing common research interests will be paired together, but all senior students can help younger students adjust to graduate school. Younger students should feel free to reach out to more senior students for advice on courses, opportunities available at Vanderbilt, and research advise/feedback.
\end{itemize}

\hypertarget{outreach-opportunities}{%
\section{Outreach Opportunities}\label{outreach-opportunities}}

There are several outreach opportunities at Vanderbilt and in the Nashville community.

\begin{itemize}
\tightlist
\item
  At Vanderbilt

  \begin{itemize}
  \tightlist
  \item
    \href{https://anchorlink.vanderbilt.edu/organization/EAN}{Vanderbilt Engineering Ambassador Network}
  \item
    \href{https://anchorlink.vanderbilt.edu/organization/bme_gsa}{Vanderbilt Biomedical Engineering Graduate Student Association}
  \item
    \href{https://medschool.vanderbilt.edu/brain-institute/new-vbi-homepage-inprogress/outreach/}{Vanderbilt Brain Institute Brain Blast}
  \item
    \href{https://www.youngscientistjournal.org/}{Vanderbilt Young Scientist Journal Peer Reviewer}
  \end{itemize}
\item
  Within the Nashville Community

  \begin{itemize}
  \tightlist
  \item
    Adventure Science Center's Scientist on Site Program
  \item
    \href{https://www.adventuresci.org/volunteer}{Adventure Science Center General Volunteer Program}
  \item
    \href{https://www.hon.org/?layoutViewMode=tablet}{Hands on Nashville}
  \end{itemize}
\end{itemize}

\hypertarget{contribute}{%
\chapter{How to Contribute to the Onboarding Document}\label{contribute}}

This onboarding document is expected to be a dynamic document that will change over time. As such, the Lippmann Lab will work as a team to keep it up-to-date. Here are a few ways that individual members can help suggest edits or create changes to the onboarding document.

\hypertarget{not-comfortable-with-coding}{%
\section{Not Comfortable With Coding}\label{not-comfortable-with-coding}}

If you are not comfortable with coding or GitHub, you can create an ``Issue'' in the GitHub repository to suggest edits to the document.

\begin{enumerate}
\def\labelenumi{\arabic{enumi}.}
\item
  Create a \href{https://github.com/}{GitHub account}.
\item
  Navigate to the \href{https://github.com/LippmannLab/Lippmann-Lab-Onboarding}{Lippmann Lab Onboarding Repository}.
\item
  Choose the ``Issues'' tab in the repository and create a new issue.
\item
  A lab member in charge of maintaining the document will look over your issue when they have time and incorporate any changes needed to the document. They may be in touch with you if you are interested in adding an additional section to the onboarding document.
\end{enumerate}

\hypertarget{comfortable-with-github}{%
\section{Comfortable with GitHub}\label{comfortable-with-github}}

If you are comfortable with using GitHub and the LaTEX/markdown language, you can suggest edits directly to the GitHub Repository.

\begin{enumerate}
\def\labelenumi{\arabic{enumi}.}
\item
  Navigate to the \href{https://github.com/LippmannLab/Lippmann-Lab-Onboarding}{Lippmann Lab Onboarding Repository}.
\item
  Select the specific chapter(s) that you want to edit. You should select the .Rmd file rather than the .html file.
\item
  Click the pencil button to start editing.
\item
  Add your edits. Add a descriptive commit message so that the person reviewing your edit knows what you changed and why. Select ``Create a new branch for this commit and start a pull request.'' Commit your changes.
\item
  A lab member who is in charge of maintaining this document will review your pull request and either discuss the change(s) with your or accept your changes.
\end{enumerate}

\hypertarget{interested-in-assisting-the-document-maintenance}{%
\section{Interested in Assisting the Document Maintenance}\label{interested-in-assisting-the-document-maintenance}}

At least one lab member will be in charge of directly editing the document. If you are interested in becoming in charge of maintaining the onboarding document, talk to Ethan or senior lab members to schedule training. No prior experience is necessary - just an interest in learning!

\end{document}
